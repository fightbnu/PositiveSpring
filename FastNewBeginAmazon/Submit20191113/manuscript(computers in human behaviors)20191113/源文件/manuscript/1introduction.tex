\section{Introduction}
\emph{Motivation}: Life is always full of ups and downs.
Accumulated stress could drain inner resources,
leading to psychological maladjustment, depression and even suicidal behaviors \citep{Nock2008Suicide}.
Compared to adults, young people exhibit high levels of stress due to their immature inner status and lack of experience \citep{older}.
According to the latest report released by the American Psychological Association in 2018,
91\% of young adults had experienced physical or emotional symptoms due to stress in the past month compared to 74\% of adults \citep{APA2018}.
More than 30 million Chinese adolescents suffer from psychological stress,
and nearly 30\% of them are at a risk of depression \citep{ChinaTeen2019}.
Stress-induced mental health problems are becoming an important social issue. 
%1.����ѹ�����������

On the other hand, positive life events, such as satisfying social interactions,
excellent academic performance and pleasant entertainment activities, could exert protective effects on emotional distress in both direct and indirect ways by '\emph{buffering}'~\citep{Shahar2002Positive, Cohen2010Positive}, with respect to physiological, psychological, and social coping resources~\citep{Cohen1984Positive,Needles1990Positive}.
Researchers indicated that positive events mitigated the relationship between negative events and maladjustment in samples of adolescents experiencing family transitions~\citep{Doyle2003Positive}.
The written expression of positive feelings could prompt increased cognitive reorganization in undergraduate students~\citep{Coolidge2009A}.
Positive events have also been linked to medical benefits, such as improved mood, serum cortisol levels, and lower levels of inflammation and hypercoagulability~\citep{Caputo1998Influence,Jain2010Effects}.
Thus, tracking the state of the stress-buffering effect is important for understanding the mental status of stressed individuals.
%2.�����¼����Ի�������ѹ����

\emph{Existing solutions}:
Previous studies have focused on measuring positive events and stress-buffering states after events through questionnaires, including the Hassles \& Uplifts Scales~\citep{Kanner1981Comparison}, the Interpretation of Positive Events Scale~\citep{Alden2008Social},
the Adolescent Self-Rating Life Events Checklist~\citep{Jun2008Influence} and the Perceived Benefit Scales~\citep{Mcmillen1998The}.
Recently, scholars have demonstrated the feasibility to sense and predict users' stress from social networks~\citep{XueUbicomp13,Xue2014Detecting,Lin2014User,Li2015When,Li2015Predicting,Li2015Using,
Li2017Analyzing,Li2017Exploring} through content (linguistic text, emoticons and pictures) and behavioral (abnormal posting time and comment/response actions) measures.

If we view the aforementioned traditional studies as static sensing of stress-buffering, this study approaches the stress-buffering as a dynamic process and aims to find a solution at both the microblogging-content and behavioral levels under the hypothesis that the occurrence of positive events can be reflected in adolescents' microblogs.
Self-report investigations are susceptible to many factors,
such as social pressure and pressure from measurement scenarios, but microblogging characteristics at the behavioral level are objective expressions that can assist in identifying content characteristics.

Another difference from the previous studies lies in that, despite the unique advantages of social networks over traditional survey methods in offering self-expressed content and behavioral information, previous microblog-based studies stopped at the analysis of stress, and none went further to capture positive events that may play a key role in adolescents' stress-coping mechanisms.
For example, ��hiking tomorrow�� might simultaneously occur and be expressed in microblogs with ��failing the exam today��.
If do not know anything about positive events, is the unilaterally detected stress the real stressful state of the current youth?
Understanding stress-buffering patterns of positive events is helpful in precisely predicting and guiding
adolescents who are coping with stress.

\emph{Our work}:
To this end,
this paper studies adolescent stress from the dual perspective of stress generation and stress-buffering
and views stress as the superposition effect of stressors and positive events.
By investigating the connection between positive events and stress changes
reflected through adolescents' microblogging content and behaviors,
we discover stress-buffering patterns of positive events and further predict future stress under such mitigation.
Exploiting stress-buffering effects of positive events
is also advantageous in handling the confusing situation
whether an adolescent who doesn't express stressful information from microblogs is actually under stress.

However, capturing the stress-buffering process of positive events is not a trivial task.
Three fundamental challenges need to be addressed:
1) What are the criteria for depicting stress-buffering effects?
2) What is the latent connection between positive events and adolescents' stress-buffering reflections in microblogs?
3) How can identify positive events and their impact interval be extracted from microblogs?

A pilot study was first conducted on the microblog data (n=27,346) of a group of high school students (n=500) associated with the school's positive scheduled events (n=75) and stressor events (n=122).
Stressful intervals were divided into two comparative categories: intervals impacted by positive scheduled events (denoted by U-SI, n=259) and intervals not impacted by positive scheduled events (denoted by SI, n=518).
After observing the posting behaviors and microblog content of the stressed students in both the SI and U-SI groups,
several implications were discussed to guide the next step of the study.
Motivated by the implications of the pilot study, we modeled the connection between positive events and adolescents' stress-buffering reflections
as the statistical difference in two comparative situations SI and U-SI.
Three groups of measures were adopted to depict adolescent stress buffering at the period level:
stress-change modes, linguistic expressions and posting behaviors.
Monotonic changes in stress intensity buffered by positive events were measured in temporal order.
As an exploration,
according to the occurrence of automatically extracted positive events,
we covered the stress-buffering effects into each time unit and integrated such an effect into the stress prediction model.

In this paper,
to automatically extract positive events,
we built upon and extended previous stress and event detection works.
A Chinese linguistic parser model was applied to extract positive events in the linguistic structure
\emph{[type, (act, doer, description)]}.
We followed the categorization of adolescents' positive events in six dimensions (entertainment, school life, romantic, peer relationships, self-cognition and family life) and extended the SC-LIWC lexicons into 2,606 phases.
Stressful intervals (SI) and stressful intervals impacted by positive events(U-SI) were identified according to their temporal order.

The rest of the paper is organized as follows.
We review the literature in section \ref{sec:related} and introduce the pilot study in section \ref{sec:obs}.
The procedure for extracting positive events is presented in section \ref{sec:frame1}.
The connection between positive events and adolescents' stress buffering from microblogs are discussed and modeled in section \ref{sec:frame2}.
We present the experimental results in section \ref{subsec:experiment} and extend the study to integrating stress-buffering patterns into future stress prediction in section \ref{subsec:predict}.
Future work is discussed in section \ref{sec:conclude}.
