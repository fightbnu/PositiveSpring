\section{Framework}
We first introduce the procedure to extract positive event and its intervals from microblogs,
Based on this,
we present a statistical model to depict
the connection between positive events and adolescents' stress-buffering patterns through three groups of content and behavioral measures.
%Finally, a none-liner time series model was proposed to combine stress-buffering patterns into future stress prediction.
\subsection{Discovery of Positive Events from Microblogs}
\label{sec:frame1}
Let $u$ = $[type,\{doer, act,$ $description\}]$ be a positive event,
where the element \emph{doer} is the subject who performs the \emph{act},
and \emph{descriptions} are the key words related to $u$.
According to psychological scales ~\citep{Jun2008Influence,hassles},
adolescent positive events mainly focus on six dimensions,
as $\mathbb{U} =\{$ 'entertainment', 'school life', 'romantic', 'pear relationship', 'self-cognition', 'family life'$\}$. We constructed our lexicon for six-dimensional positive events from two sources.
The basic positive words are selected from the psychological lexicon C-LIWC (expectation, joy, love, surprise)~\citep{Tausczik2010The}.
Then we built six topic lexicons by expanding basic positive words from adolescent microblogs,
containing 452 phrases in 'entertainment',
273 phrases in 'school life',
138 phrases in 'romantic',
91 phrases in 'peer relationship',
299 phrases in 'self-recognition' and 184 phrases in 'family life', with totally 2,606 phrases,
as examples shown in table \ref{tab:topicWords}.
Additionally, we labeled \emph{doer} words (i.e., \emph{teacher}, \emph{mother}, \emph{I, we}) in positive lexicons.

\begin{table*}
\centering
\caption{\small{Topic words of six-dimensional positive events.}}
\label{tab:topicWords}
\small{
\begin{tabular}{lll}
\toprule
dimension & example words & total \\ \midrule
entertainment  & hike, travel, celebrate, dance, swimming, ticket, shopping, air ticket, theatre, party, Karaoke,& 452\\
                      & self-driving tour, game, idol, concert, movie, show, opera, baseball, running, fitness, exercise & \\
school life    & reward, come on, progress, scholarship,admission, winner, diligent, first place, superior & 273\\
				      & hardworking, full mark,  praise, goal, courage, progress, advance, honor, collective honor& \\
romantic       &  beloved, favor, guard, anniversary,  concern, tender, deep feeling, care, true love, promise, & 138\\
				      & cherish, kiss, embrace, dating, reluctant, honey, sweetheart, swear, love, everlasting, goddess &\\
pear relation  & listener, company, pour out, make friends with, friendship, intimate, partner, team-mate, brotherhood& 91\\
self-cognition & realize, achieve, applause, fight, exceed, faith, confidence, belief, positive, active, purposeful & 299\\
family life    & harmony, filial, reunite, expecting, responsible, longevity, affable, amiability, family, duty & 184\\
\bottomrule
\end{tabular}}
\end{table*}

\subsubsection{Linguistic Parser Model}
Positive events were identified through Chinese natural language processing platform \citep{Che2010}.
For each post, after word segmentation, we parsed each sentence to find its linguistic structure,
and then matched the main linguistic components with positive topic lexicons in each dimension.
The linguistic parser model was applied to identify the central verb of current sentence, namely the \emph{act}.
It constructed the relationship between the central verb and corresponding \emph{doer} and \emph{description} elements.
By searching these elements in positive topic lexicons,
the existence of positive events were identified.
Due to the sparsity of posts, the element \emph{act} might be empty.
\emph{Descriptions} were collected by searching all nouns, adjectives and adverbs.
Examples of positive events extracted from adolescents' microblogs are listed in table \ref{tab:uplifts}.
For the post 'Thanks all my dear friends hosting the party. Happiest birthday!!!',
it was processed as \emph{doer='friends', act = 'expecting', description = 'party'},
and \emph{type = 'entertainment'}.

\begin{table}
\begin{center}
\caption{\small{Extracted positive events from microblogs.}}
\small{
\begin{tabular}{|l|} \hline
I am really looking forward to the spring outing on Sunday now. \\
(doer:\emph{I}, act:\emph{looking forward}, description:\emph{spring outing})\\\hline
My holiday is finally coming [smile]. \\
(doer:\emph{My holiday}, act:\emph{coming}, description:\emph{[smile]})\\\hline
First place in my lovely math exam!!! In memory of it.\\
(description:\emph{first place, math, exam, memory})\\\hline
You are always here for me like sunshine. \\
(doer:\emph{You}, description:\emph{sunshine})\\\hline
Thanks all my dear friends hosting the party.
Happiest birthday!!!\\
(doer:\emph{friends}, act:\emph{thanks}, description:\emph{party, birthday})\\\hline
I know my mom is the one who support me forever, no matter \\
when and where. (doer:\emph{mom}, act:\emph{support})\\ \hline
Expecting Tomorrow' Adult Ceremony[Smile][Smile]~~\\
(act: \emph{expecting}, description:\emph{Adult Ceremony})\\\hline
\end{tabular}}
\label{tab:uplifts}
\end{center}
\end{table}

\subsubsection{Impact Intervals of Positive Events}
We followed and extended ~\citep{Li2017Analyzing} to identify the impact interval of each positive event to further study its stress-buffering pattern.
Splitting interval is a common time series problem, and here we identified the target interval in three steps.

Step1:
Extracted positive events, stressor events and filtered out candidate intervals.
For each candidate interval,
we set its length to more than 3 days and a maximum gap of 1 day between two neighboured stressed days.
Since the stress series detected from microblogs were discrete points,
loess method was adopted to highlight characteristics of the stress curve.

Step2: Judged stressful intervals through hypothesis testing.
A Poisson based probability model was adopted to measure how confidently the current interval was a stressful interval.
Here the stressful posting rates under stress $\lambda_1$ and normal conditions $\lambda_0$ were modeled as two independent poisson process:
\begin{equation}
Pr[N=n|\lambda_i]=\frac{e^{-\lambda_i T}{(\lambda_i T)}^n}{n!}
\end{equation}
where $i\in\{0,1\}$, $n=0,1,\cdots,\infty$.
We expected that $\lambda_1 > \lambda_0$, and measured the probability as $P(\lambda_1>\lambda_0|N_1, T_1, N_0, T_0)$,
where $N_1, N_0$ are the number of stressful posts, and $T_1, T_0$ are time duration corresponding to $\lambda_1$ and $\lambda_0$.
Without loss of generality, we assume a Jeffreys non-informative prior on $\lambda_1$ and $\lambda_0$,
and inferred the posterior distribution $P(\lambda_1|N_1)$ and $P(\lambda_0|N_0)$ according to Bayes Rule.
Thus for current interval $I_1$ and historical normal interval $I_0$,
the quantified probability $\beta = P(\lambda_1>\lambda_0|I_1,I_0)$ $\in (0,1)$ indicated the confidence
whether $I_1$ was a stressful interval.

Step 3: Divided stressful intervals into SI set and U-SI set in temporal order.
For a detected stressful interval $I = <t_1,\cdots,t_n>$, we considered the temporal order between $I$ and any detected positive event $u$ happening at time point $t_u$ in three cases:
1) If the positive event $u$ happened during the stressful interval, i.e., $t_u \in [t_1,t_n]$, the positive interval $I$ was judged as $I \in U-SI$.
2) If the positive event happened nearby a stressful interval,
considering the probability that it conducted impact on current stressful interval.
Here the gap between $t_u$ and $I$ is limited to $\xi$, i.e.,
if $t_u \in [t_{1}-\xi, t_1)\cup(t_{n},t_{n}+\xi]$, then $I \in U-SI$.
If a stressful interval satisfies none of the above conditions, we classify it into the SI set.
3) Other stressful intervals were divided into U-SI set.

%M1: �������measure���㣬�ɴ˻���3��or4��
%M2: �Ƿ��б�Ҫ�ϲ�֮ǰ����ϸ������(��׸��)
\subsection{Relationship Between Positive Events and Adolescents' Stress-buffering from Microblogs}
\label{sec:frame2}
%��ǰ���ӽ�
%�ӽ��
%Linguistic expressions
\subsubsection{Measures}
\paragraph{Topic}
Positive and stressful expressions were extracted from the post content.
The first linguistic measure was the frequency of \emph{positive word},
which represented the positive emotion in current interval.
The second measure was the frequency of \emph{positive event topic words} in six dimensions,
reflecting the existence of positive events.
\citep{Li2014Major} showed that self-mentioned words showed high probability that the current stressor event was related to the author,
rather than the opinion about a public event or life events about others.
Another important factor was wether existing \emph{self-mentioned words} (i.e., \emph{'I','we','my'}).
Except positive-related linguistic descriptions, we also took stressful linguistic characters as measures,
while also offered information from the complementary perspective.
The frequency of \emph{stressor event topic words} in five dimensions represented the degree of attention for each type of stressor event.
The frequency of \emph{pressure words} reflected the degree of general stress emotion during the interval.
\paragraph{Positive and Stressful Emotions}
%new
\paragraph{Posting behaviors}
Stress could lead to abnormal posting behaviors,
reflecting user's changes in social engagement activity ~\citep{Liang2015Teenagers}.
In this study,
we considered four measures of posting behaviors in each time unit (day),
and presented each measure as a corresponding series.
The first measure was \emph{posting frequency},
representing the total number of posts per day.
Research in \cite{Li2017Analyzing} indicated that overwhelmed adolescents tended to post more to express their stress for releasing
and seeking comfort from friends.
The second measure \emph{stressful posting frequency} per day
was based on existing stress detection result and highlights the stressful posts among all posts.
The third measure was the \emph{positive posting frequency}, indicating the number of positive posts per day.
The forth measure \emph{original frequency} was the number of original posts, which filters out re-tweet and shared posts.
Compared to forwarded posts, original posts indicated higher probability that users were talking about themselves.
Thus in each interval, user's posting behavior was represented as a four-dimension vector.

\paragraph{Stress change mode}
The global stress change mode during a stressful interval was depicted through four measures:
\emph{sequential stress level, length, RMS,} and \emph{peak}.
Basically, \emph{stress level} per day constructed a sequential measure during a stressful interval,
recording stress values and fluctuation on each time point.
As positive events might conduct impact on stressed adolescents,
and postpone the beginning or promote the end of a stressful interval,
we took \emph{length} as the second factor representing the interval stress change mode.
To quantify the intensity of stress fluctuations,
\emph{RMS} (root mean square) of stress values through the interval was adopted  as the third measure.
\emph{Peak} value was adopted as the forth measure to show the maximal stress value in current interval.
Next,
based on the above measures,
we quantified the difference between SI and U-SI sets, thus to track the stress-buffering pattern of positive events.


\subsubsection{Statistical Model of Stress-buffering Effects}
In our problem,
there were two sets of stressful intervals to compare:
the SI set and the U-SI set,
containing stressful intervals not affected by positive events
and stressful intervals impacted by positive events, respectively.
The basic elements in each set were stressful intervals.
Each interval was modeled as a multi-dimensional vector according to the three groups of measures in section ~\ref{measures}.
Thus we formulated this comparison problem as finding the correlation between the two sets of multi-dimension points.
Specifically, we adopted the multivariate two-sample hypothesis testing method
\cite{Li2017Correlating,Johnson2012Applied} to model such correlation.
In this two-sample hypothesis test problem,
the basic idea is judging whether the multi-dimension points (i.e., stressful intervals)
in set SI and set U-SI were under different statistical distribution.
Assuming the data points in SI and U-SI were randomly sampled from distribution $F$ and $G$, respectively,
then the hypothesis was denoted as:
\begin{equation}
H_0: F = G \quad versus \quad H_1: F \neq G.
\end{equation}

Under such hypothesis,
$H_0$ indicates points in SI and U-SI were under similar distribution,
while $H_1$ means points in SI and U-SI were under statistically different distributions,
namely positive events conducted obvious stress-buffering effect on current user.
Since each point in the two sets (SI and U-SI) was depicted in multi-dimensions,
here we took the KNN (K-Nearest Neighbor) \cite{Schilling1986Multivariate}
based method to judge the existence of significant difference between SI and U-SI.
For simplify, we used the symbol $A_1$ to represent set SI,
and $A_2$ represent set U-SI.
In the KNN algorithm,
for each point $\ell_{x}$ in the two sets $A_1$ and $A_2$,
we expected its nearest neighbors (\emph{the most similar points}) belonging to the same set of $\ell_x$.
The model derivation process was presented in ~\ref{mod:mod1}.

%%%--------------------
For each interval, three groups of behavioral measures are considered: \emph{posting behavior},
\emph{stress change mode} and \emph{linguistic expressions},
indicated as \bm{${<D_p}$},\bm{${D_s}$},\bm{${D_l>}$}, respectively.
To measure the correlation for each group of measures,
the Euclidean distance is adopted to calculate the distance of structured points in $A_1$ and $A_2$.

For each point $\ell x \in A=A_1\bigcup A_2$,
let $NN_r(\ell_x,A)$ be the function to find the $r-$th nearest neighbor of $\ell_x$.
Specifically, three sub-functions of $NN_r(.)$ are defined as $PNN_r(.)$, $SNN_r(.)$ and $LNN_r(.)$,
corresponding to user's posting behaviors, stress change mode and linguistic expressions in each stressful interval, respectively.

For point $\ell_x$ with posting behavior matrix \bm{${D_p^x}$}, stress change mode matrix \bm{${D_s^x}$},
and linguistic expression matrix \bm{${D_l^x}$},
the $r$-th nearest neighbor of $\ell_x$ in each measure is denoted as:
\begin{equation}
\begin{aligned}
& PNN_r(\ell_x,A)
= \{y | min\{||\textbf{D}_p^x-\textbf{D}_p^y ||_2\}, y\in(A/\ell_x)\} &\\
& SNN_r(\ell_x,A)
= \{z | min\{||\textbf{D}_s^x-\textbf{D}_s^z ||_2\}, z\in(A/\ell_x)\} \\
& LNN_r(\ell_x,A)
= \{w | min\{||\textbf{D}_l^x-\textbf{D}_l^w ||_2\}, w\in(A/\ell_x)\} &
 \end{aligned}
 \end{equation}
The $r$-th nearest neighbor considering all three groups of measures is denoted as:
\begin{align}
&NN_r(\ell_x,A) = \{v | min\{a \times ||\textbf{D}_p^x-\textbf{D}_p^v||_2+\\
&b \times ||\textbf{D}_s^x-\textbf{D}_s^v||_2+
c \times ||\textbf{D}_l^x-\textbf{D}_l^v||_2\}, v\in(A/\ell_x) \}
\end{align}
In this study, we set $a = b = c = 1/3$.
Next, let $I_r(\ell_x,A1,A2)$ be the function denoting whether the $r$-th nearest neighbor is in the same set with $\ell_x$:
\begin{equation}
I_r(\ell_x,A_1,A_2) =
\left\{ \begin{array}{ll}
1, \quad if \ell_x \in A_i  \&\& NN_r(\ell_x,A)\in A_i,\\
0, \quad otherwise
\end{array}
\right.
\end{equation}
Let $T_{r,n}$ denote the proportion that pairs containing two points from the same set among all pairs formed by $\ell_x \in A$
and its $k$ nearest neighbors:
\begin{equation}
T_{k,n}= \frac{1}{n\times k}\sum_{i=1}^{n}\sum_{j=1}^{k}I_j(x,A_1,A_2)
\end{equation}
The value of $T_{k,n}$ shows how differently the points in the two testing sets (SI and U-SI) perform in three groups of measures.
If the value of $T_{r,n}$ is close to $1$,
it can be shown that the two underlying distributions $F$ and $G$ for $SI$ and U-SI are significantly different,
indicating current positive events conduct obvious restoring impact on the teens' stress series.
Let $\lambda_1=|A_1|$ and $\lambda_2=|A_2|$, the statistic value $Z$ is denoted as:
\begin{align}
&Z=(nr)^{1/2}(T_{r,n}-\mu_{r})/\sigma_{r}\\
&\mu_r=(\lambda_1)^2+(\lambda_2)^2\\
&{\sigma_r}^2=\lambda_1\lambda_2+4{\lambda_1}^2{\lambda_2}^2
\end{align}
where $\mu_r$ is the expectation and ${\sigma_r}^2$ is the variance of $Z$.
Based on hypothesis test theory \cite{Johnson2012Applied},
when the size of the testing set ($\lambda_1$ and $\lambda_2$) are large enough,
$Z$ obeys a standard Gaussian distribution.

Thus we judge whether the positive events have conducted significant restoring impact on the teen's stress series as follows:
if $f(SI,USI)=(nr)^{1/2}(T_{r,n}-\mu_{r})/{\mu_r}^2>\alpha$ ($\alpha = 1.96$ for $P=0.025$),
then the hypothesis $H_1$ is true.

\subsubsection{Monotonous Model of Stress-buffering}
To verify the monotonous stress changes at both the early and late stress-buffering stages,
for each stressful interval in SI (n=2,582) and U-SI (n=1,914),
we compared its stress intensity with the front and rear adjacent intervals using t-test method.

For a stressful interval $I = <t_i,t_{i+1},\cdots,t_j>$,
let $I^{front} = <t_m,\cdots,t_{i-1}>$ be the adjacent interval before $I$,
and $I^{rear} = <t_{j+1},\cdots,t_n>$ be the rear adjacent interval of $I$.
The length of $I^{front}$ and $I^{rear}$ are set to $|I|$.
For the set of stressful intervals $SI$ composed of $<I_1,I_2,\cdots,I_N>$,
the corresponding sets of adjacent front and rear intervals are denoted as $SI^{front}$ and $SI^{rear}$.
Similarly, for the set of stressful intervals $USI$ = $<UI_1,UI_2,\cdots, UI_M>$ impacted by positive events,
the corresponding sets of adjacent front and rear intervals are denoted as $USI^{front}$ and $USI^{rear}$.
We compare the intensity of stress changes in following four situations,
where $g(.)$ is the function comparing two sets: \\
1) $g(SI,SI^{front}$) returns if intensive change happens when stressful intervals begin.\\
2) $g(SI,SI^{rear}$) returns if stress changes intensively after the stressful intervals end.\\
3) $g(USI,USI^{front}$) returns if intensive change happens when stressful intervals affected by positive events appears.\\
4) $g(USI,USI^{rear}$) returns if stress changes intensively after stressful intervals affected by positive events end.

In our problem, taking the comparison between $SI$ and $SI^{rear}$ for example,
the basic computation element $I_k \in SI \cup SI^{rear}$ in both sets is a multi-dimension interval.
Here we adopt the t-test method as the intensity computation function $g(.)$.
%The t-test algorithm measures if intensive positive or negative monotonous correlation
%exists between two sample sets.
The function $g(.) = t_{score}$ $\in$ (-1,1) is represented as:

\begin{equation}
\small{g(SI,SI^{rear})}= \frac{\mu_{SI}-\mu_{SI^{rear}}}{\sqrt{\frac{(n_1-1)\sigma^2_{SI}+(n_2-1)\sigma^2_{SI^{rear}}}{n_1+n_2-2}(\frac{1}{n_1}-\frac{1}{n_2})}}
\end{equation}
where $\mu_{SI}$ and $\mu_{SI^{rear}}$ are the mean stress values of intervals in sets $SI$ and $SI^{rear}$,
and $\sigma_{SI}$ and $\sigma_{SI^{rear}}$ are the variance stress values of intervals in sets $SI$ and $SI^{rear}$, respectively.
If $g(SI,SI^{rear})$ $> \alpha$, stress intensity in $SI^{rear}$ show significant decrease compared with $SI$ (monotonic negative effect).
If $g(SI^{front},SI)$ $< -\alpha$, stress intensity in $SI$ show significant increase compared with $SI^{front}$ (monotonic positive effect).
Here we adopt $\alpha$ = 1.96, $P$ = 0.025.
We conduct comparison for above four situations,
to observe whether the occurrence of positive events relieve the monotonic negative effect of $g(SI,SI^{rear})$
and the monotonic positive effect of $g(SI^{front},SI)$.
