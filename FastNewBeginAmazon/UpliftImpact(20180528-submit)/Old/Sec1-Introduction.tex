\IEEEraisesectionheading{\section{Introduction}\label{sec:case study}}
\textbf{Motivation.}
Stress is an important health issue.
77\% of adults reported regularly experiencing physical symptoms caused by stress,
and an estimated 75\% to 90\% of visits to primary care physicians were stress-related~\cite{Hoffmann2015}.
Teens are more stressed out than adults.
According to an annual survey published by the American Psychological Association~\cite{Aleccia2014},
teens routinely said that their school-year stress levels were far higher than they thought
was healthy and their average reported stress exceeded that of adults.
Around 30\% of teens reported feeling sad or depressed because of stress and 31\%
felt overwhelmed. Another 36\% said that stress makes them tired and 23\% said they had
skipped meals because of it.
On average, teens reported their stress level was 5.8 on 10-point scale, compared with 5.1 for adults.
Stress seems to be getting worse for some teens, according to the survey.
About 31\% of kids said their stress level had increased in the past year,
twice as many as those who said it went down, and 34\% said they expected their stress level would rise in the coming year.
There is no doubt that with the fast speed of society and high expectation from adults and themselves,
teens will become more stressful than ever before.

Stress is the way human body responds to challenges.
The causes of stress are known as \emph{stressors} and there are literally hundreds of different types of stressors.
Any event that one finds threatening and difficult to cope with can be a potential cause of stress.
Typical stressor events
that may cause youth stress
include schoolwork and career decisions,
physical and cognitive changes of puberty, family and peer conflicts,
crammed schedules, dating and friendships, and so on.
The ways in which teens cope with these stressor events can
have significant short- and long-term consequences on their physical and
emotional health. Difficulties in handling stress can lead to mental health
problems, such as depression, anxiety disorders, and even suicide~\cite{Events2013}.
Due to the immaturity and lack of life experiences,
teenagers may not always take effective measures to deal with the stress.
Thus, timely discovery and effective intervention prevention of
adolescents�� mental disease is necessary and urgent.

\textbf{Existing solutions.}
Signs of teens stress could show up in their behaviors, emotions, and bodies, and can thus be
analyzed and discovered through various psychological signals
(e.g. heart rate variability HRV~\cite{mental}, electrocardiogram ECG~\cite{activity},
galvanic skin response GSR~\cite{fuzzy}, blood pressures, and electromyogram~\cite{driver})
and physical signals (e.g. gestures, voice, eye gaze, facial expressions, pupil dilation and
blink rates~\cite{phone1,Ubicomp2014}).
Recent studies demonstrated the feasibility of stress detection
through the open social media
microblogs (e.g. Twitter, Tencent Weibo, and Sina Weibo)~\cite{XueUbicomp13,HIS,EDBT2014,LinHuiJie14,LHJ,Zhao2015}.
Through teen's posting content (linguistic text, emoticons, and repetitive punctuation), posting time,
and social interaction with friends (being liked/reposted/cared and comment-response acts) underneath each post,
teens stress category (academy, inter-personnel, self-cognition, or affection) and stress level
could be detected.

If we view the aforementioned approaches as \emph{symptom}-based \emph{point}-wise exploration towards teens stress detection,
this study approaches the problem from the \emph{cause} of the stress, and aims at a
\emph{period}-wise detection solution under the assumption that
teen's stress is caused by one or more stressor events and will last for a period of time.
Short-lasting acute stress is not considered in the current study,
as long-lasting stress has a more serious impact on teens physical and psychological health than acute stress
due to sustained high levels of chemicals released in the response.
From this perspective, stressful periods are more worth paying attention to than stressful points.

Another difference from the previous work is that,
despite the unique advantage of microblogs over the sensory approaches in offering linguistic textual information for analysis,
the aforementioned approach stopped at detecting the general stress category,
and none went further to extract specific events that may cause the stress.
From example, it is \emph{quarreling with friends} that leads to teen's
\emph{inter-personnel} stress. Knowing stressor event details is helpful
in understanding and guiding stressful teens towards healthy ways of thinking and reacting.

\textbf{Our work.}
To this end, this paper pushes forward the problem of teens psychological stress detection from emotional symptom-oriented level to event-oriented level.
By investigating the correlation between stressor events and stress reflected through a series of posts on Tencent Weibo (http://t.qq.com)~\cite{tencent},
we discover teen's stressful periods, and further extract details of stressor events that trigger the stress.
Exploiting the correlation between stressor events and stress for stressful period detection
is also advantageous in handling data uncertainty when a teen's post is missing and
false alarms of intermittent stressful points.

However, detection of stressful periods and stressor events from the microblog is not a trivial task.
Three fundamental questions need to be addressed:
(1) \emph{what is the latent correlation between stressor events and teens stressful reflection on the microblog?}
(2) \emph{what is counted as a teen's stressful period?} and
(3) \emph{what are the criteria for teen's stressor events extraction?}

In this paper, we stand upon and extend the previous stressful posts detection result.
Motivated by the empirical findings from a user study at a high school that
\emph{the rate of teens posting stressful posts is higher
during stressor-event periods than non-stressor-event periods,}
we propose to model the correlation between stressor events and teens stressful periods
through a statistic measure, which is the
Bayesian posterior probability of teens stressful posting rates
during event and non-event periods.
As this probability only depends on stressful post rates rather than on
specific stressful post time points during
event and non-event periods, the model is attractive for statistic
inference in large-scale systems in terms of both measurement and computation.
With the model, we can analyze and identify maximal stressful periods, from which
a set of stressor events are extracted.

In the study, we follow the categorization of teens stressor events,
containing daily hassles and life events,
that world-wide teens likely encounter from five \emph{dimensions}, namely, \emph{school life, family life, peer
relation, romantic relation}, and \emph{self-cognition}~\cite{hassles,scale1,scale2,scale3}.
Each dimension contains a set of items, representing \emph{event types}, as listed in Table~\ref{tab:stressorEvents}.
A stressor event is presented in the form of
\texttt{[Dimension, Event-type, Event-instance (description, doer, act, object, time, location)]}, \texttt{influence)},
where \texttt{Dimension} and \texttt{Event-type} are from Table~\ref{tab:stressorEvents},
and \texttt{Event-instance} refers to the time-specific event occurrence with the
involved role, act, object, time, location, linguistic description, and its influence
measurement upon stress.

Due to teens causal and informal expression on the microblog,
it may be hard to identify and distinguish concrete event instances with complete occurring time and roles
information. Therefore, we generalize and present the extracted stressor events based on common dimensions and event types.
Considering that a stressor event tends to stimulate the increase of a teen's stress level,
we rank the stressor events at the level of instance, type, and dimension based on their negative stress impact that,
the higher the stress impact is, the greater teen's stress level is increased, the longer the teen's stress will sustain,
and the more the teen may post about.
The experimental study in a high school with 124 participants
shows that our method achieves promising performance in detecting stressful periods
(\emph{recall} 0.761, \emph{precision} 0.737, and $F_1$-\emph{measure} 0.734) and stressor events
\emph{recall} 0.763, \emph{precision} 0.756, and $F_1$-\emph{measure} 0.759).
We compare the performance of our stressor event detection approach with the
state-of-art top-1 personal life event detection approach,
achieving 13.72\% higher in \emph{precision}, 19.18\% higher in \emph{recall}, and 16.50\% higher in $F_1$-\emph{measure}.


\setlength{\tabcolsep}{2.5pt}
\begin{table}
\begin{footnotesize}
\begin{center}
\begin{tabular}{|l|l|} \hline
\textbf{Dimension} & \textbf{Event-Type}  \\ \hline
\emph{School}  & Having to study things you do not understand \\ \cline{2-2}
\emph{Life}  & Feeling tired of study \\	\cline{2-2}
(25 types) &	Keeping up with schoolwork	\\ \cline{2-2}
&	Having too much homework \\ \cline{2-2}
& Having exams or tests \\ \cline{2-2}
& Having difficulty in some subjects	\\ \cline{2-2}
& Teachers expecting too much from you \\	\cline{2-2}
& Getting along badly with teachers \\	\cline{2-2}
&  Lack of respect from teachers	 \\ \cline{2-2}
& Being scolded by teachers	\\ \cline{2-2}
& Being punished corporally  \\  \cline{2-2}
&	Getting up early in the morning to go to school	\\ \cline{2-2}
& Entering a higher school	\\ \cline{2-2}
&	Going to school	\\  \cline{2-2}
& Failing to be admitted to a university	\\  \cline{2-2}
&	Not getting enough time for leisure/fun \\ \cline{2-2}
& Worrying about losing behind \\ \cline{2-2}
&	Wasting time	\\ \cline{2-2}
& Unsatisfying exam results	\\  \cline{2-2}
&	Too many interruptions	\\  \cline{2-2}
& Failing to reach expectations	\\  \cline{2-2}
&	Unsatisfying teaching quality	\\  \cline{2-2}
& Failing in awards competition	\\  \cline{2-2}
& Being suspended from school	\\  \cline{2-2}
&	Changing class or school	\\ \hline

\emph{Family}  &  Having bad relationship with family members \\  \cline{2-2}
\emph{Life}  & Being beaten/scolded by parents \\  \cline{2-2}
(17 types) &	Parents working away from home for a long time	\\  \cline{2-2}
& Parents hassling you about the way you look	\\  \cline{2-2}
& Household disasters	\\  \cline{2-2}
& Disagreements between you and parents \\  \cline{2-2}
&	Serious disease of family member  \\ \cline{2-2} %or friends	\\
& Lack of understanding by your parents	\\  \cline{2-2}
&	Death of family member \\ \cline{2-2} %or friends	\\
& Not being taken seriously by your parents	\\ \cline{2-2}
& Parents expecting too much from you	\\  \cline{2-2}
&	Not having enough money to buy the things you want	\\  \cline{2-2}
&	Family having financial difficulties	\\  \cline{2-2}
& Bad habits of parents	 \\  \cline{2-2}
&	Having to take on new financial responsibilities \\ \cline{2-2} %when growing older	\\
& Parents' divorce \\  \cline{2-2}
& Parents' quarreling 	\\ \hline

\emph{Peer}  &  Being hassled for not fitting in with peers \\  \cline{2-2}
\emph{Relation} & Trying to keep up with the Joneses \\  \cline{2-2}
(6 types) & Cannot bear the roommate's behaviors	\\  \cline{2-2}
& Rumor or satire from peers	\\  \cline{2-2}
& Misunderstood by others	\\  \cline{2-2}
& Fierce competition among peers	\\ \hline

\emph{Self-} &  Unsatisfied with appearance/weight \\  \cline{2-2}
\emph{Cognition} & Having to make decisions about future work or education	\\  \cline{2-2}
(15 types) & Bad sleep conditions	\\  \cline{2-2}
&	Having to take on new family responsibilities \\ \cline{2-2} %when growing older	\\
& Change of daily routine	\\  \cline{2-2}
&	Doubting the meaning of life	\\  \cline{2-2}
& Getting sick or hurt	\\  \cline{2-2}
&	Depression	\\  \cline{2-2}
& Troubles of menstruation	\\  \cline{2-2}
&	Regretting for past behaviors	\\  \cline{2-2}
& Internet addition	\\  \cline{2-2}
&	Losing face in public	\\  \cline{2-2}
& Worrying about future	\\  \cline{2-2}
&	Being frightened/threatened unexpectedly	\\  \cline{2-2}
& Losing things or being stolen	 		\\ \hline

\emph{Romantic} &  Secret adoration \\  \cline{2-2}
\emph{Relation} & Sexual behavior \\  \cline{2-2}
(6 types) & Getting along with your boy/girl-friend  \\  \cline{2-2}
& Breaking up with your boy/girl-friend	 \\  \cline{2-2}
& Too curious about sex  \\  \cline{2-2}
&	Being ignored or rejected by the person you want to \\
& go out with	\\ \hline
\end{tabular}
\caption{Typical adolescent stressor event types}
\label{tab:stressorEvents}
\end{center}
\end{footnotesize}
\end{table}

%\hl{In this paper, we define the new problem of identifying teens stressful periods and stressor events from the microblog.}

To our knowledge, this is the first attempt in the literature that
analyzes and identifies teens stressful periods and stressor events from the microblog.
The poisson-based probability model for the correlation between stressor events and stressful posting behaviors,
together with the event-oriented stress detection method proposed in this paper
targets at solving the psychological health problem for adolescents from appearance to essence,
thus enabling targeted guidance and support from
teens parents, teachers, peers, and mental health care professionals in stress coping.

The reminder of the paper is organized as follows. We review some closely related work in Section 2.
We empirically study teens abnormal posting behaviors incurred by stressor events in Section 3.
A microblog-based framework for inferring maximal stressful periods and extracting stressor events is then
presented in Section 4.
%and statistically model the correlation between stressor events and stressful periods
%A formal problem statement is given in Section 5, followed by the method
%given in Section 6.
We report our performance study in Section 5, and
discuss implications of the study in Section 6.
We conclude the paper in Section 7.

\comment{
The impact of a stressor event can last for a period of time,
depending on its stimulus attribute (i.e., property, frequency, intensity, etc.) and personal psychological endurance,
affected by age and life experience~\cite{Cohen1983A}.

The aim of this study is to discover these stressor events (SE) that could cause teen's stress
from teens social behaviors on microblog.

On the basis of the psychological health research~\cite{science1976,Events2013},
we categorize four type of stressor events corresponding to
the four stress categories (i.e., \emph{academy, inter-personnel, self-cognition,} and \emph{affection}),
and list them in Table~\ref{tab:events}.

A stressor is a chemical or biological agent, environmental condition, external stimulus or an event that causes stress to an organism.[1]

An event that triggers the stress response may include:
environmental stressors (hypo or hyper-thermic temperatures, elevated sound levels, over-illumination, overcrowding)
daily stress events (e.g., traffic, lost keys, quality and quantity of physical activity)
life changes (e.g., divorce, bereavement)
workplace stressors (e.g., high job demand vs. low job control, repeated or sustained exertions, forceful exertions, extreme postures)
chemical stressors (e.g., tobacco, alcohol, drugs )
social stressor (e.g., societal and family demands)

In psychology, events that trigger one's stress response are called
\emph{stressors}, which can be categorized as life events, environmental factors, workplace stressors,
chemical stressors, and social stressors~\cite{Stressor}.

As the major kind of stressor, the close relation between stressful life events and human's psychological health has been extensively studied since 1930s~\cite{science1976}.
Stressful life events have been implicated in the development of a range of mental disorders, including mood and anxiety disorders\cite{wikiMental}\cite{Events2013}.

Teenagers, like adults, may experience stress everyday, and can benefit from learning stress management skills.

Stress in teenagers is common nowadays, so recognizing stressor events and help them reduce stress are important
guide them towards helpful ways of thinking and healthy lifestyle choices.

For chronic stress, parents or caring adults can help teens understand the cause
of the stress and then identify and practice
positive ways to manage the situation.

Stress in teenagers �C and anyone �C isn��t necessarily a bad thing.

Stress is the way your body responds to challenges and gets you ready to face them with attention,
energy and strength. Stress gets you ready for action. When you feel you can cope with these challenges,
stress gives you the motivation to get things done.

But there can be serious problems when your stress is greater than your ability to cope.

Stress is the physical, mental and emotional human response to a particular stimulus, otherwise called as 'stressor'.


Things that can cause youth stress include school pressure and career decisions, after-school or summer jobs,
dating and friendships, pressure to wear certain types of
clothing, jewelry, or hairstyles
Pressure to experiment with drugs,
alcohol, or sex
Pressure to be a particular size or
body shape. With girls, the focus
is often weight. With boys, it is
usually a certain muscular or
athletic physique.
Dealing with the physical and
cognitive changes of puberty
Family and peer conflicts
Being bullied or exposed to
violence or sexual harassment
Crammed schedules, juggling
school, sports, after-school
activities, social life,
and family obligations

Events triggering human's stress response are defined as \emph{stressors} in psychology, categorized into life events, environmental factors, workplace stressors, chemical stressors and social stressors~\cite{Stressor}.
As the major kind of stressor, the close relation between stressful life events and human's psychological health has been extensively studied since 1930s~\cite{science1976}.
Stressful life events have been implicated in the development of a range of mental disorders, including mood and anxiety disorders\cite{wikiMental}\cite{Events2013}.

The impact of a stressful event can last for a period of time,
depending on its stimulus attribute (i.e. property, frequency, intensity, etc.) and personal psychological endurance,
affected by age and life experience~\cite{Cohen1983A}.
People's response to the stressor, defined as \emph{coping mechanism} in psychology, is an important element in stress theory~\cite{coping}.

Through abundant tweets expressed by adolescents naturally and truly,
it is promising to detect life events triggering stress at source automatically and timely,
thus to assist further prevention of mental disease for adolescents from a new perspective.

}
