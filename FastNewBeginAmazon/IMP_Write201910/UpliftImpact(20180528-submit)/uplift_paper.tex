\documentclass[10pt, journal]{IEEEtran}%,compsoc
\usepackage{balance}  % to better equalize the last page
\usepackage{graphics} % for EPS, load graphicx instead
\usepackage{times}    % comment if you want LaTeX's default font
\usepackage{url}      % llt: nicely formatted URLs

\usepackage{graphicx}
\usepackage{subfigure}
\usepackage{subfigure}
\usepackage{helvet}
\usepackage{courier}
\usepackage{diagbox}
\usepackage{amsmath}
\usepackage{multirow}
\usepackage{booktabs}
\usepackage{makecell}
\usepackage{amssymb}
\usepackage{threeparttable}
\usepackage[justification=centering]{caption}
\usepackage[linesnumbered,ruled,vlined,commentsnumbered]{algorithm2e}
\usepackage{color}
\usepackage{xcolor}
%\usepackage{multirow}
%\usepackage[table]{xcolor}
\usepackage{colortbl}
\usepackage{float}
\usepackage{bm}
\usepackage[normalem]{ulem} % use normalem to protect \emph
\newcommand\hl{\bgroup\markoverwith
  {\textcolor{yellow}{\rule[-.5ex]{2pt}{2.5ex}}}\ULon}

\usepackage{Cite}

\ifCLASSINFOpdf

\else

\fi

\hyphenation{op-tical net-works semi-conduc-tor}

\usepackage{hyperref}
\usepackage{breakurl}
\newtheorem{definition}{Definition}
\def\boxend{\hspace*{\fill} $\Box$}
\newcommand{\comment}[1]{}
\renewcommand{\multirowsetup}{\centering}

\begin{document}

\title{Analyzing the Restoring Impact of Uplift Events on Overwhelmed Teenagers via Microblogs}

\author{Qi~Li,~Yuanyuan~Xue,~Liang~Zhao~and~Ling~Feng~\IEEEmembership{Senior~Member,~IEEE}
\IEEEcompsocitemizethanks{\IEEEcompsocthanksitem
\hspace{-0.3cm}$^1$Dept. of Computer Science and Technology,
Centre for Computational Mental Healthcare Research, Institute of Data Science,
Tsinghua University, Beijing, China;\protect\\
$^2$Institute of Social Psychology, Xi'an Jiaotong University, Xi'an, China.\protect\\
E-mail: \{liqi13,xue-yy12\}@mails.tsinghua.edu.cn,\{fengling\}@tsinghua.edu.cn,
\{zhaoliang0415\}@xjtu.edu.cn,
}}

\IEEEtitleabstractindextext{
\begin{abstract}
As an important concept in psychological theory,
restoring is an essential process in human's stress coping system.
Timely and efficient restoring of stress could help teenagers get out of overwhelmed status as soon as possible.
Previous research has explored the possibility of detecting teenagers' stress series and
mining the impact of stressor events from social media.
On the contrary, the research on auto-analyzing the restoring ability of uplift events still calls for more exploration,
due to the uncertainty and complexity of various restoring situations.
In this paper,
we give a deep inside into the stress easing function of uplift events on the real data set of 124 high school students.
A two-sample based statistical model is conducted to analyze the stressful behavioral correlations
when uplift events happened to overwhelmed students from multiple perspectives.
Experimental results show that our method could measure the restoring impact of school scheduled uplift events with 69.52\% accuracy, 
and integrating such impact of uplift events helps reduce the stress prediction errors efficiently (with MSE, RMSE, MAPE and MAD reduced by 49.34\%, 28.81\%, 26.80\% and 42.11\%, respectively). 
Our exploration provides guidance for school and parents that which kind of uplift events could help relieve students' overwhelmed stress in both stress prevention and stress early stopping situations.
\end{abstract}

\begin{IEEEkeywords}
uplift event, stressor event, teenagers, microblogs
\end{IEEEkeywords}
}

\maketitle.

\IEEEdisplaynontitleabstractindextext

\IEEEpeerreviewmaketitle

\section{Introduction}
Life is always full of ups and downs.
According to the transactional model of stress \cite{Susan1984Stress}, our stress mainly comes from daily hassles.
The cumulative stress caused by the small and frequent stressful life events could drain people's inner resources,
leading to psychological maladjustment, ranging from depression to suicidal behaviours \cite{Nock2008Suicide}.
As a global public health concern,
suicide caused by psychological stress has become the second leading cause of death among young adults in college
(Centers for Disease Control and Prevention, 2012).

On the other hand, positive life events (called \emph{uplifts} in psychological theory) such as satisfying social interactions,
excellent academic performance and pleasant entertainment activities are conceptualized in psychological literature as exerting a protective effect on emotional distress \cite{Cohen1984Positive} \cite{Cohen2010Positive} \cite{Needles1990Positive}.
Compared with adults, young people exhibit more exposure to uplift events, as well as hassles,
due to the immature inner status and lack of experience \cite{older}.
Researchers indicate that positive events mitigate the relation between negative events and maladjustment in samples of adolescents experiencing family transitions \cite{Doyle2003Positive}.
The written expression of positive feelings has also be shown to prompt increased cognitive re-organization among an undergraduate student group \cite{Coolidge2009A}.

Positive uplifts can not only help reinforce adolescents' sense of well-being,
and help restore the capacity for dealing with stress,
but also have been linked to medical benefits, such as improving mood, serum cortisol levels, and lower levels of inflammation and hypercoagulability \cite{Jain2010Effects}.
Through examining the relationship between self-reported positive life events and blood pressure (BP) in 69 sixth graders,
researchers found that  increased perceptions of positive life events might act as a buffer to elevated BP in adolescents
\cite{Caputo1998Influence}.

The protective effect of uplift events is hypothesized to operate in two ways:
directly and indirectly by 'buffering' \cite{Cohen2010Positive}.
In the direct way,
the more positive uplift events people experienced, the less distress they experience.
While in the indirectly way, positive life events play its role by buffering the effects of negative events on distress.
A pioneer experiment conducted by Reich and Zautra provided enlightening evidence for us \cite{Shahar2002Positive}.
In this experiment, sampled college students who reported initial negative events were encouraged to engage in either two or twelve pleasant activities during one-month, and compared with students in the controlled group experiencing no pleasant activities.
Results indicated that participants in the two experimental groups reported greater quality of life compared with controlled students,
and participants who engaged in twelve uplift events exhibited lower stress compared with whom engaging two or none uplifts,
implicating the protective effect of uplift events on adolescents.

Previous exploration for the protective effect of uplift events on adolescents are mostly conducted in psychological area,
relying on traditional manpower-driven investigation and questionnaire.
The pioneer psychological researches provide us valuable implications and hypothesis,
while limited by labor cost, data scale and single questionnaire based method.
With the high development of social networks,
today adolescents tend to express themselves and communicate with outside world through posting microblogs,
at anytime and anywhere.
The self-motivated expressions could deliver much information about their inner thoughts and life styles.
In recent years, some research on psychological stress analysis based on social network has emerged,
from basically detecting stress intensity from microblog content
\cite{XueUbicomp13,Xue2014Detecting}, predicting future stress level in time series
\cite{Li2015Predicting,Li2015When,Li2015Using,Li2017Exploring},
to extracting stressor events and stressful intervals \cite{Li2017Analyzing}.
These researches explored applying psychological theories into social network based stress mining,
offering effective tools for adolescent stress sensing.
Nevertheless, few work takes an insight into the restoring function of uplift events,
which plays an important role opposite to stress,
as the essential way for adolescent psychological stress easing.

In this paper,
we aim to continually mine the restoring impact of uplift events leveraging abundant data source from microblogs,
to further provide guidance for school and parents that when and which kind of uplift events could help relieve students' overwhelmed stress in both stress prevention and stress early stopping situations.
To model such a practical application problem, several challenges exist.

\begin{itemize}
\item \emph{How to extract uplift events from microblogs and identify corresponding impact interval}?
The impact of uplift events is highlighted when the teen is under stress, with various relative temporal order.
Extracting such scenarios from teen's messy microblogs is the first and basic challenge for further analysis.
\item \emph{How to qualitatively and quantitatively measure the restoring impact conducted by uplift events}?
There are multiple clues related to teens' behaviours from microblogs, i.e.,
depressive linguistic content, abnormal posting behaviours.
The teen might act differently under similar stressful situations when the uplift event happens or not.
It is challenging to find such hidden correlation between uplift events and teen's behavioural characters.
\end{itemize}
Moreover, for different types of uplift events, the restoring impact might be different.
And for each individual, the protective and buffering effect for stress might also varies according to the personality.
All these questions guide us to solve the problem step by step.

In this paper, we first conduct a case study on real data set
to observe the posting behaviours and contents of stressful teens under the influence of uplift events.
We conduct the case study on the real data set of 124 high school students associated with the school's scheduled uplift and stressor event list.
Several observations are conducted to guide the next step research.
Next, we extract uplift events and the corresponding impacted interval from microblogs.
We define and extract structural uplift events from posts using linguistic parser model based
on six-dimensional uplift scale and LIWC lexicons.
Independent stressful intervals (SI) and stressful intervals impacted by uplifts (U-SI) are extracted considering temporal orders.
To quantify the restoring impact of uplift events,
we describe a teen's stressful behaviours in three groups of measures (stress intensity, posting behaviour, linguistic),
%to uniform
and model the impact of uplift events as the statistical difference between the sets of SI and U-SI in two aspects:
the two-sample based method is employed for variation detection,
and the t-test correlation is conducted to judge the monotonous correlation.

%to be add. 升华一下:意义是什么?

The rest of the paper is organized as follows.
We introduce related works in section \ref{sec:related}, 
and conduct the data observation in section \ref{sec:obs}.
The preliminaries and problem formulation are presented in section \ref{sec:problem}.
We conduct the procedure for extracting uplift events and identifying the impact interval in section \ref{sec:interval},
and introduce the detailed method for analyzing the restoring impact of uplift events in section 
\ref{sec:impact}. 
We present the experimental results in section \ref{sec:experiment}, 
and discuss the future work in section \ref{sec:conclude}.
\section{Related Work}
\label{sec:related}
\subsection{Protective function of uplift events}
Many psychological researchers have focused on the restorative function of positive events and emotions with respect to physiological, psychological, and social coping resources.
Folkman \emph{et al.}\cite{Folkman2010Stress} identified three classes of coping mechanisms that are associated with positive emotion during chronic stress: positive reappraisal, problem-focused coping, and the creation of positive events.
The author also considered the possible roles of positive emotions in the stress process, and incorporated positive emotion into a revision of stress and coping theory in
the work \cite{Folkman1997Positive}.
They conducted a longitudinal study of the care giving partners of men with AIDS and described coping processes that were associated with positive psychological states in the context of intense distress.
Cohen \emph{et al.} \cite{Cohen2010Positive} hypothesized that protective effect of uplift events operates in both directly (i.e., more positive uplift events people experienced, the less distress they experience) and indirectly ways by 'buffering'.

Chang \emph{et al.} \cite{Chang2015Loneliness} investigated the protective effect of positive events in a sample of 327 adults, and found that the positive association between loneliness and psychological maladjustment was found to be weaker for those who experienced a high number of positive life events, as opposed to those who experienced a low number of positive life events.
This is assistant with the conclusion made by Kleiman \emph{et al.}
\cite{Evan2014Social} that positive events act as protective factors against suicide individually and synergistically when they co-occur, by buffering the link between important individual differences risk variables and maladjustment.
Through exploring naturally occurring daily stressors, Ong \emph{et al.}
\cite{Ong2006Psychological} found that over time,
the experience of positive emotions functions to assist high-resilient individuals to recover effectively from daily stress.
In the survey made by Santos \emph{et al.} \cite{Santos2013The}, strategies of positive psychology are checked as potentially tools for the prophylaxis and treatment of depression, helping to reduce symptoms and for prevention of relapses.
Through a three-week longitudinal study, Bono \emph{et al.}
\cite{Bono2013Building} examined the correlation between employee stress and health and positive life events, and concluded that naturally occurring positive events are correlated with decreased stress and improved health.

\subsection{Measuring the Impact of Uplift Events}
To measure the impact of uplift events,
Doyle \emph{et al.} \cite{Kanner1981Comparison} conducted \emph{Hassles and Uplifts Scales},
and concluded that the assessment of daily hassles and uplifts might be a better approach to the prediction of adaptational outcomes than the usual life events approach.
Silva \emph{et al.} \cite{Silva2008The} presented the \emph{Hassles \& Uplifts Scale} to assess the reaction to minor every-day events in order to detect subtle mood swings and predict psychological symptoms.
To measure negative interpretations of positive social events,
Alden \emph{et al.} \cite{Alden2008Social} proposed the interpretation of positive events scale (\emph{IPES}), and analyzed the relationship between social interaction anxiety and the tendency to interpret positive social events in a threat-maintaining manner.
Mcmillen \emph{et al.} \cite{Mcmillen1998The} proposed the \emph{Perceived Benefit Scales} as the new measures of self-reported positive life changes after traumatic stressors, including lifestyle changes, material gain, increases in selfefficacy, family closeness, community closeness, faith in people, compassion, and spirituality.
Specific for college students,
Jun-Sheng \emph{et al.} \cite{Jun2008Influence} investigated in 282 college students using the \emph{Adolescent Self-Rating Life Events Checklist}, and found that the training of positive coping style is of great benefit to improve the mental health of students.

\subsection{Analyzing adolescent stress from social media}
With the high development of social network,
researchers tend to digging user' psychological status from the self-expressed public data source.
Billions of people record their life, share multi-media content, and communicate with friends through such platforms, e.g.,
Tencent Microblog, Twitter, Facebook and so on.
Inspired by rich microblogging content,
Xue \emph{et al.} \cite{XueUbicomp13, Xue2014Detecting} proposed to detect adolescent stress from single microblog utilizing machine learning methods by extracting stressful topic words, abnormal posting time, and interactions with friends.
Lin \emph{et al.} \cite{Lin2014User} construct a deep neural network to combine the high-dimensional picture semantic information into stress detecting.
Based on the stress detecting result,
Li \emph{et al.} \cite{Li2015Predicting}\cite{Li2015Using}\cite{Li2015When} adopted a series of multi-variant time series prediction techniques (i.e., Candlestick Charts, fuzzy Candlestick line and  SVARIMA model) to predict the future stress trend and wave.
Taking the linguistic information into consideration,
Li \emph{et al.} \cite{Li2017Exploring} employed a NARX neural network to predict a teen's future stress level referred to the impact of co-experiencing stressor events of similar companions.
All above pioneer work focused on the generation and development of teens' stress, providing solid basic techniques for broader stress-motivated research from social networks.

To find the source of teens' stress, previous work \cite{Li2017Analyzing} developed a frame work to extract stressor events from microblogging content and filter out stressful intervals based on teens' stressful posting rate.
Based on such research background, this paper starts from a completely new perspective, and focuses on the buffering effect of positive events on restoring stress.
Thus we push forward the study from how to find stress to the next more meaningful stage: how to deal with stress.

\subsection{Correlation analysis for multivariate time series}
Basic correlation analysis methods on time series focused on univariate data have been well studied.
As the most widely adopted method,
the Pearson correlation analysis \cite{Cohen1988Statistical} measures the linear correlation between two variables $X$ and $Y$.
One inevitable defect is that Pearson correlation is too sensitive to outlier values.
To overcome such drawback,
Spearman Rank correlation \cite{C1987The}
and Kendall Rank correlation \cite{Mcleod2011Kendall}
are proposed based on Pearson correlation.
While Pearson correlation estimates linear relationships,
Spearman correlation estimates monotonic relationships (whether linear or not),
and are calculated as the Pearson correlation between the rank values of two variables.
The Kendall correlation mainly assesses the similarity of the orderings of the data when ranked by each of the quantities.
The above correlation methods are usually used to estimate relationship between single-dimensional variables,
and cannot be adopted directly in our microblog content based scenario.

For multivariate time series analysis, two-sample based methods are widely adopted.
Such kind of methods are deduced to check whether two samples come from the same underlying distribution, which is assumed to be statistically unknown.
Correspondingly, various kernel
\cite{Sch2006A} and distance-based methods \cite{Schilling1986Multivariate}
(e.g., the nearest neighbor based method two-sample method) are proposed.
Scholkopf \emph{et al.} \cite{Sch2006A} proposed to transform the distance between two variables and nearest neighbors into a reproducing kernel Hilbert space (RKHS), and solve the problem using Maximum Mean Discrepancy.
In work \cite{Schilling1986Multivariate},
Schilling \emph{et al.} adopted the $r$-nearest neighbor based method to partition two set of event driven time series data.
The global proportion of the right divided neighbors are calculated to estimate whether there exists statistically difference between the two sets.
We use the $r$-nearest neighbor based two-sample method in our problem, thus to measure the distance and correlation between two multi-dimension variables.

\begin{table}
\begin{center}
\caption{Examples of uplift events expressed and extracted from teens' microblogs.}
\begin{tabular}{l} \hline \rowcolor{gray!40}
I am really looking forward to the spring outing on Sunday now. \\ \rowcolor{gray!40}
(Doer:\emph{I}, Act:\emph{looking forward}, Object:\emph{spring outing})\\
My holiday is finally coming [smile]. \\
(Doer:\emph{My holiday}, Act:\emph{coming}, Object:\emph{[smile]})\\ \rowcolor{gray!40}%\hline
First place in my lovely math exam!!! In memory of it.\\ \rowcolor{gray!40}
Object:\emph{first place, math, exam, memory})\\ %\hline
You are always here for me like sunshine. \\
(Doer:\emph{You}, Object:\emph{sunshine})\\ \rowcolor{gray!40} %\hline
Thanks all my dear friends to take the party for me. Happiest birthday!\\ \rowcolor{gray!40}
(Doer:\emph{friends}, Act:\emph{thanks}, Object:\emph{party, birthday})\\
Be yourself. Trust yourself and follow your heart. \\
(Doer:\emph{yourself}, Act:\emph{trust}, Object:\emph{heart})\\ \rowcolor{gray!40} %\hline
Feel proud of our play in the Games. Our class is always the family!!!\\ \rowcolor{gray!40}
(Doer:\emph{Our}, Object:\emph{class, family})\\
A good film always makes bring comfort and happiness to me.\\
(Doer:\emph{me}, Act:\emph{bring}, Object:\emph{comfort, happiness})\\ \rowcolor{gray!40}%\hline
I know my mom is the one who support me forever, no matter \\ \rowcolor{gray!40}
when and where. (Doer:\emph{mom}, Act:\emph{support})\\ \hline
\end{tabular}
\label{tab:uplifts}
\end{center}
\end{table}

\begin{table}
\begin{center}
\caption{Examples of stressor events expressed and extracted from teens' microblogs.}
\begin{tabular}{l} \hline \rowcolor{gray!40}
I don't know how long can I bear the nag.\\ \rowcolor{gray!40}
(Doer:\emph{I}, Act:\emph{bear}, Object:\emph{nag})\\ %\hline
Parents like to judge everything around me with their emotion.
\\(Doer:\emph{parents}, Act:\emph{judge}, Object:\emph{everything})\\ \rowcolor{gray!40}%\hline
Hope that my uncle could revive earlier.\\ \rowcolor{gray!40}
(Doer:\emph{my uncle}, Act:\emph{revive})\\%\hline
Every one betrayed me.
\\(Doer:\emph{every one}, Act:\emph{betray}, Object:\emph{me})\\ \rowcolor{gray!40} %\hline
I'm too weak to handle such a fierce competition.\\ \rowcolor{gray!40}
(Doer:\emph{I}, Act:\emph{too weak to handle}, Object:\emph{competition})\\%\hline
I just felt hurt, depressed, self-abased and sad.
\\(Doer:\emph{I}, Act:\emph{feel hurt, depressed, self-abased and sad})\\ \rowcolor{gray!40}%\hline
My holiday is filled with all kinds of homework.\\ \rowcolor{gray!40}
(Doer:\emph{My holiday}, Act:\emph{fill with}, Object:\emph{homework})\\ %\hline
Unescapably, it's time to go back to school.
\\(Act:\emph{go back}, Object:\emph{school})\\ \rowcolor{gray!40} %\hline
When can you be aware of my heart-broken feeling again and again?\\ \rowcolor{gray!40}
(Doer:\emph{you}, Act:\emph{be aware of}, Object:\emph{heart-broken feeling})\\ \hline
\end{tabular}
\label{tab:stressors}
\end{center}
\end{table}

\begin{table}
\centering
\caption{Examples of school scheduled uplift events and stressor events.}
\label{tab:example}
\begin{tabular}{cccc}
\toprule
Type & Date	& Work Content	& Grade	\\
\midrule
\emph{stressor event} & 2014/4/16 & \emph{first day of mid-term exam} & grade1,2\\
\emph{uplift event} & 2014/11/5 & \emph{campus art festival} & grade1,2,3\\
\bottomrule
\end{tabular}
\end{table}

\begin{figure}
\centering
\caption{Examples of school related stressor events, uplift events and a student's stress fluctuation}
\includegraphics[width=\linewidth]{figs/exampleWave.eps}
\label{fig:example}
\end{figure}

\section{Data Observation}
\label{sec:obs}
We built our dataset based on two sources: 1) the microblogs of students coming from Taicang High School,
collected from January 1st, 2012 to February 1st, 2015;
and 2) list of scheduled school events, with exact start and end time.
We filtered out 124 active students according to their posting frequency from over 500 students,
and collected their microblogs throughout the whole high school career. Totally 29,232 microblogs are collected in this research,
where 236 microblogs per student on average, 1,387 microblogs maximally and 104 posts minimally.

\emph{Uplift events and stressor events}.
The list of weekly scheduled school events (from February 1st, 2012 to August 1st 2017) are collected from the school's official website
\footnote{http://stg.tcedu.com.cn/col/col82722/index.html}, with detailed event description and grade involved in the event.
There are 122 stressor events and 75 uplift events in total.
Here we give the examples of scheduled uplift and stressor events in high school life, as shown in Table~\ref{tab:example}.
There are 2-3 stressor events and 1-2 uplift event scheduled per month.


\emph{Stress detected from microblogs}.
Since our target is to observe the restoring impact of uplift events for teenagers under stress.
Based on previous research~\cite{XueUbicomp13},
we detected the stress level (ranging from 0 to 5) for each post;
and for each student, we aggregated the stress during each day by calculating the average stress of all posts.
The positive level (0-5) of each post is identified based on the frequency of positive words (see Section 5 for details).
Figure~\ref{fig:example} shows three examples of a student's stress fluctuation during three mid-term exams,
where the uplift event \emph{campus art festival} was scheduled ahead of the first exam,
the uplift event \emph{holiday} happened after the second exam,
and no scheduled uplift event was found nearby the third exam.
The current student exhibited differently in above three situations, with the stress lasting for different length and with different intensity.

To further observe the influence of uplift events for students facing stressor events,
we statistic all the stressful intervals~\cite{Li2017Analyzing} detected surround the scheduled examinations over the 124 students during their high school career.
For each student, we divide all his/her stressful intervals into two sets:
1) stressful intervals under the influence of neighbouring uplift events (e.g., \emph{Halloween activity}), and 2) independent stressful intervals.
Figure~\ref{fig:frequency} shows five measures of each student during the above two conditions:
the \emph{accumulated stress}, the \emph{average stress} (per day), the \emph{length of stressful intervals},
the \emph{frequency of academic topic words}, and the \emph{ratio of academic stress among all types of stress}.
For each measure, we calculate the average value over all eligible slides for each student.

\emph{Findings}. Comparing each measure in scheduled exam slides under the two situations: 1) existing neighbouring uplift events or 2) no neighbouring scheduled uplift events,
we find that students during exams with neighbouring uplift events exhibit less average stress intensity (both on accumulated stress and average stress),
and the length of stress slides are relatively shorter.
Further, we statistic the frequency of academic related topic words for each exam slide (as listed in Table \ref{tab:studyWords}),
and look into the ratio of academic stress among all five types of stress.
Results in Figure~\ref{fig:frequency} shows that most students talked less about the upcoming or just-finished exams when uplift events happened nearby,
with lower frequency and lower ratio.
The stress intensity and type distribution detected from each student's microblogs varies due to personal life experience, posting habits and express styles.
The statistic result shows clues about the stress-relieving ability of scheduled uplift events,
and thus helps shape our problem as how to quantify the influence of uplift events,
thus to provide further guidance for planning campus activities to help relive high school students' psychological stress effectively.

\begin{table}[h]
\centering
\caption{Examples of academic related topic words.}
\label{tab:studyWords}
\begin{tabular}{c}
\toprule
exam, fail, review, score, grade, test paper, rank, pass, math, chemistry\\
homework, recite, regress, fall behind, tension, stressed out, physics,\\
nervous, mistake, answer, question, puzzle, difficult, lesson, careless, \\
\bottomrule
\end{tabular}
\end{table}


\begin{figure}
\centering
\caption{Compare students' stress during exam intervals in two situations:
1) affected by neighboring uplift events (U-SI), 2) no uplift events occurred nearby (SI)}
\includegraphics[width=\linewidth]{figs/frequency.eps}
\label{fig:frequency}
\end{figure}


\input{Sec-Problem}
\input{Sec-Interval}
\input{Sec-Impact}
\input{Sec-Experiment}
\section{Conclusion}
\label{sec:conclude}
In this paper, 
we give a deep inside into the stress easing function of uplift events on the real data set of 124 high school students.
A two-sample based statistical model is conducted to analyze the stressful behavioral correlations
when uplift events happened to overwhelmed students from multiple perspectives.
Experimental results show that our method could measure the restoring impact of school scheduled uplift events efficiently, and integrating the impact of uplift events helps reduce the stress prediction errors.
Our research provides guidance for school and parents that
which kind of uplift events could help relieve students' overwhelmed stress
in both stress prevention and stress early stopping situations.
Our future work will focus on digging the overlap impact of multiple uplift events in more complex situations, as well as the frequent appearing patterns of different types of uplift events and stressor events,
thus to provide more accurate analysis and restoring guidance for individual teenagers.
%\section*{Acknowledgments}
%here acknowledgments

\bibliographystyle{IEEEtran}
\bibliography{reference-LiQi} %IEEEabrv,sample
\end{document}


