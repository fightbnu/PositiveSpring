\section{Discussion}
A microblog-based framework towards the detection of
adolescent psychological stressful periods and stressor events is presented.
The underlying poisson-based probability model for the correlation between stressor events and
stressful posting behaviors,
together with the event-oriented stress detection method proposed in this paper,
targets at sensing teens' psychological stressful periods and stressor events
from appearance to essence.

\textbf{Advantage and Disadvantage of the Probability-based Model.}
As this probability model only depends on stressful tweeting rates rather than on
specific stressful tweeting time during event and non-event periods,
the model is attractive for statistic inference in both measurement and computation.
The shortcoming of the model is that
it needs teen's stressful tweeting rates $\lambda_0$
during non-stressor-event periods from historic training data.
Also, the method relies on a confidence threshold
$\tau$ to distinguish between a stressful period and a non-stressful period.
When $\tau$ is around 0.5, we can obtain the best performance in this study.
However, when a teen has no historic data to derive appropriate values of $\lambda_0$ and $\tau$,
we may have to pre-set the values by consulting with
other teens with similar backgrounds and characteristics,
and dynamically tune the values based on the feedback.
Here, accurately identifying similar teens in stress detection is critical to
the overall performance.


\textbf{Extension to Other Microblogs.} Although the current study is carried on Chinese posts on Tencent Weibo,
the general framework (consisting of identifying stressful periods and then
extracting stressor events) is applicable to other microblogs like Twitter, Tumblr, and Sino Weibo.
However, for microblogs in different languages,
different linguistic processing techniques are
needed in building stress-related lexicons,
sensing teens stressful emotion from each post, and finally
extracting details of stressor events.
