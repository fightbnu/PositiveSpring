\section{Conclusion}
In this paper, we present techniques to analyze and identify teens stressful periods and stressor events from the microblog.
A poisson-based probability approach is developed to
model the correlation between stressor events and stressful posting behaviors.
With the model, we discover maximal stressful periods
and extract stressor events from each stressful period.
We examine 124 high-school students' posts from January 1, 2012 to February 1, 2015, and
detect stressful periods with
(\emph{recall} 0.761, \emph{precision} 0.737, and $F_1$-\emph{measure} 0.734),
and top-3 stressor events with (\emph{recall} 0.763, \emph{precision} 0.756, and $F_1$-\emph{measure} 0.759).
The performance of our top-1 stressor event detection is better than
the state-of-art personal life event detection approach, which is about
13.72\% higher in \emph{precision}, 19.18\% higher in \emph{recall}, and 16.50\% higher in $F_1$-\emph{measure}.

Due to the complexity of stressor theory, it can be said that identifying stressor events and incurred stressful periods is interesting and challenging, and much more
related issues need further investigation.
First, people's response to the stressor, defined as coping mechanism in psychology, is also an important element in stress theory.
We will take coping mechanisms into consideration as a key factor in our future work.
Second, while this study shows a promising direction to detect stressful periods and stressor events for adolescents,
the microblog is still an insufficient source with a lot of data missing.
Integrating multiple sources (e.g., GPS trajectory outlier data and sleep outlier data)
could be helpful to provide more comprehensive and continuous information,
enabling further improvement of the detection performance.



