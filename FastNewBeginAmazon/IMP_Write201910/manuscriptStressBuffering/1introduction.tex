\section{Introduction}
\emph{Motivation}: Life is always full of ups and downs.
Accumulated stress could drain inner resources,
leading to psychological maladjustment, depression and even suicidal behaviours \citep{Nock2008Suicide}.
Compared with adults, young people exhibit more exposure to stress
due to the immature inner status and lack of experience \citep{older}.
According to the latest report released by American Psychological Association in 2018,
91\% of youngest adults had experienced physical or emotional symptoms due to stress in the past month compared to 74\% of adults \citep{APA2018}.
More than 30 million Chinese adolescents are suffering from psychological stress,
and nearly 30\% of them have a risk of depression \citep{ChinaTeen2019}.
Stress-induced mental health problems are becoming an important social issue worldwide.
%1.����ѹ�����������

%2.�����¼����Ի�������ѹ����
On the other hand, positive life events, such as satisfying social interactions,
excellent academic performance and pleasant entertainment activities,
could exert protective effects on emotional distress in both directly and indirectly ways by \emph{'buffering'} ~\citep{Shahar2002Positive, Cohen2010Positive},
with respect to physiological, psychological, and social coping resources~\citep{Cohen1984Positive,Needles1990Positive}.
Researchers indicated that positive events mitigated the relation between negative events and maladjustment in samples of adolescents experiencing family transitions \citep{Doyle2003Positive}.
The written expression of positive feelings had also been proven to prompt increased cognitive re-organization among an undergraduate student group \citep{Coolidge2009A}.
Positive events also have been linked to medical benefits, such as improving mood, serum cortisol levels, and lower levels of inflammation and hypercoagulability \citep{Caputo1998Influence,Jain2010Effects}.
Thus, tracking the state of stress-buffering is important for understanding the mental status of stressed individuals.
%------------

\emph{Existing solutions}:
Previous studies have been focusing on conducting measurement of positive events and stress-buffering state after events through questionnaires,
including Hassles \& Uplifts Scales~~\citep{Kanner1981Comparison},
Perceived Benefit Scales~~~~~\citep{Mcmillen1998The},
Interpretation of Positive Events Scale~\citep{Alden2008Social}
and Adolescent Self-Rating Life Events Checklist~~~\citep{Jun2008Influence}.
Recent scholars have demonstrated the feasibility to sense and predict users' stress from social networks~\citep{XueUbicomp13,Xue2014Detecting,Lin2014User,Li2015When,Li2015Predicting,Li2015Using,
Li2017Analyzing,Li2017Exploring}
through content (linguistic text, emoticons, pictures)
and behavioral (abnormal posting time, comment/response actions) measures.

If we view the aforementioned traditional studies on positive events as static sensing of stress-buffering,
this study approaches the problem to the dynamic process of stress-buffering
and aims at a solution at both microblogging content and behavioral levels under the hypothesis 
that the occurrence of positive events can be reflected in adolescents' microblogs. 
Since the subjective self-report investigations are susceptible to many factors,
such as social appreciation and pressures from measurement scenarios,
microblogging characteristics at the behavioral level are objective expressions that can assist content characteristics. 

Another difference from the previous studies lies in that,
despite the unique advantages of social networks over traditional survey methods in offering self-expressed content and behavioral information,
previous microblog-based researches stopped at the analysis of stress, 
and none went further to capture positive events that may play a key role in adolescents' stress coping mechanism. 
For example, it is ��hiking tomorrow�� that might simultaneously occur and be expressed in microblogs with ��losing the exam today��. 
If we couldn't know anything about positive events, 
is the stress unilaterally detected is the real stressful state of the current youth? 
Understanding stress-buffering patterns of positive events
is helpful in precisely predicting and guiding stressful adolescents coping with stress.

\emph{Our work}:
To this end,
this paper proposes to study adolescent stress in a dual perspective of stress generation and stress-buffering, 
and view it as the superposition effect of stressors and positive events. 
By investigating the connection between positive events and stress changes
reflected through adolescents' microblogging content and behaviors,
we discover stress-buffering patterns of positive events and further predict future stress under such mitigation.
Exploiting stress-buffering effects of positive events
is also advantageous in handling the confusing situation
whether an adolescent who doesn't express stressful information from microblogs is actually under stress.

However, capturing the stress-buffering process of positive events is not a trivial task.
Three fundamental challenges need to be addressed: 
1) What are the criteria to depict stress-buffering? 
2) What is the latent connection between positive events and adolescents' stress-buffering reflections from microblogs? 
3) How to extract positive events and its impact interval from microblogs? 

%1
A pilot study was firstly conducted on the microblog data (n=27,346) of a group of high school students (n=500)
associated with the school's scheduled positive events (n=75) and stressor events (n=122).
Stressful intervals were divided into two comparative categories:
intervals impacted by scheduled positive events (denoted as U-SI, n=259)
and intervals not impacted by scheduled positive events (denoted as SI, n=518).
After observing the posting behaviors and contents of stressed students in both SI and U-SI groups,
several implications were discussed to guide the next step study.

%3
Motivated by the implications from the pilot study,
we modeled the connection between positive events and adolescents' stress-buffering reflections
as the statistical difference in two comparative situations SI and U-SI.
Three groups of measures were adopted to depict adolescent stress-buffering at period-level:
stress change modes, linguistic expressions and posting behaviours.
Positive events buffered monotonous changes of stress intensity
Monotonous changes of stress intensity buffered by positive events were measured in temporal order.
%4
As an exploration,
according to the occurrence of automatically extracted positive events,
we covered its stress-buffering effects into each time unit
and integrated such effect into stress prediction.

%2
In this paper,
to realize automatically extraction of positive events,
we stood upon and extended previous stress and event detection works.
A Chinese linguistic parser model was applied to extract positive events in the linguistic structure
\emph{[type, (act, doer, description)]}.
We followed the categorization of adolescents' positive events in six dimensions (entertainment, school life, romantic, pear relationship, self-cognition, family life) and extended SC-LIWC lexicons to 2,606 phases.
Stressful intervals (SI) and stressful intervals impacted by positive events(U-SI) were identified according to temporal orders.

The rest of the paper is organized as follows.
We review the literature in section \ref{sec:related},
and introduce the pilot study in section \ref{sec:obs}.
The procedure for extracting positive events is presented in section \ref{sec:frame1}.
The connection between positive events and adolescents' stress-buffering from microblogs are discussed and modeled in section \ref{sec:frame2}.
We present the experimental results in section \ref{subsec:experiment},
and extend the study to integrating stress-buffering patterns into future stress prediction in section \ref{subsec:predict}.
Future work is discussed in section \ref{sec:conclude}.
