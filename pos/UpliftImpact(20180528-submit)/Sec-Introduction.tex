\section{Introduction}
Life is always full of ups and downs.
According to the transactional model of stress \cite{Susan1984Stress}, our stress mainly comes from daily hassles.
The cumulative stress caused by the small and frequent stressful life events could drain people's inner resources,
leading to psychological maladjustment, ranging from depression to suicidal behaviours \cite{Nock2008Suicide}.
As a global public health concern,
suicide caused by psychological stress has become the second leading cause of death among young adults in college
(Centers for Disease Control and Prevention, 2012).

On the other hand, positive life events (called \emph{uplifts} in psychological theory) such as satisfying social interactions,
excellent academic performance and pleasant entertainment activities are conceptualized in psychological literature as exerting a protective effect on emotional distress \cite{Cohen1984Positive} \cite{Cohen2010Positive} \cite{Needles1990Positive}.
Compared with adults, young people exhibit more exposure to uplift events, as well as hassles,
due to the immature inner status and lack of experience \cite{older}.
Researchers indicate that positive events mitigate the relation between negative events and maladjustment in samples of adolescents experiencing family transitions \cite{Doyle2003Positive}.
The written expression of positive feelings has also be shown to prompt increased cognitive re-organization among an undergraduate student group \cite{Coolidge2009A}.

Positive uplifts can not only help reinforce adolescents' sense of well-being,
and help restore the capacity for dealing with stress,
but also have been linked to medical benefits, such as improving mood, serum cortisol levels, and lower levels of inflammation and hypercoagulability \cite{Jain2010Effects}.
Through examining the relationship between self-reported positive life events and blood pressure (BP) in 69 sixth graders,
researchers found that  increased perceptions of positive life events might act as a buffer to elevated BP in adolescents
\cite{Caputo1998Influence}.

The protective effect of uplift events is hypothesized to operate in two ways:
directly and indirectly by 'buffering' \cite{Cohen2010Positive}.
In the direct way,
the more positive uplift events people experienced, the less distress they experience.
While in the indirectly way, positive life events play its role by buffering the effects of negative events on distress.
A pioneer experiment conducted by Reich and Zautra provided enlightening evidence for us \cite{Shahar2002Positive}.
In this experiment, sampled college students who reported initial negative events were encouraged to engage in either two or twelve pleasant activities during one-month, and compared with students in the controlled group experiencing no pleasant activities.
Results indicated that participants in the two experimental groups reported greater quality of life compared with controlled students,
and participants who engaged in twelve uplift events exhibited lower stress compared with whom engaging two or none uplifts,
implicating the protective effect of uplift events on adolescents.

Previous exploration for the protective effect of uplift events on adolescents are mostly conducted in psychological area,
relying on traditional manpower-driven investigation and questionnaire.
The pioneer psychological researches provide us valuable implications and hypothesis,
while limited by labor cost, data scale and single questionnaire based method.
With the high development of social networks,
today adolescents tend to express themselves and communicate with outside world through posting microblogs,
at anytime and anywhere.
The self-motivated expressions could deliver much information about their inner thoughts and life styles.
In recent years, some research on psychological stress analysis based on social network has emerged,
from basically detecting stress intensity from microblog content
\cite{XueUbicomp13,Xue2014Detecting}, predicting future stress level in time series
\cite{Li2015Predicting,Li2015When,Li2015Using,Li2017Exploring},
to extracting stressor events and stressful intervals \cite{Li2017Analyzing}.
These researches explored applying psychological theories into social network based stress mining,
offering effective tools for adolescent stress sensing.
Nevertheless, few work takes an insight into the restoring function of uplift events,
which plays an important role opposite to stress,
as the essential way for adolescent psychological stress easing.

In this paper,
we aim to continually mine the restoring impact of uplift events leveraging abundant data source from microblogs,
to further provide guidance for school and parents that when and which kind of uplift events could help relieve students' overwhelmed stress in both stress prevention and stress early stopping situations.
To model such a practical application problem, several challenges exist.

\begin{itemize}
\item \emph{How to extract uplift events from microblogs and identify corresponding impact interval}?
The impact of uplift events is highlighted when the teen is under stress, with various relative temporal order.
Extracting such scenarios from teen's messy microblogs is the first and basic challenge for further analysis.
\item \emph{How to qualitatively and quantitatively measure the restoring impact conducted by uplift events}?
There are multiple clues related to teens' behaviours from microblogs, i.e.,
depressive linguistic content, abnormal posting behaviours.
The teen might act differently under similar stressful situations when the uplift event happens or not.
It is challenging to find such hidden correlation between uplift events and teen's behavioural characters.
\end{itemize}
Moreover, for different types of uplift events, the restoring impact might be different.
And for each individual, the protective and buffering effect for stress might also varies according to the personality.
All these questions guide us to solve the problem step by step.

In this paper, we first conduct a case study on real data set
to observe the posting behaviours and contents of stressful teens under the influence of uplift events.
We conduct the case study on the real data set of 124 high school students associated with the school's scheduled uplift and stressor event list.
Several observations are conducted to guide the next step research.
Next, we extract uplift events and the corresponding impacted interval from microblogs.
We define and extract structural uplift events from posts using linguistic parser model based
on six-dimensional uplift scale and LIWC lexicons.
Independent stressful intervals (SI) and stressful intervals impacted by uplifts (U-SI) are extracted considering temporal orders.
To quantify the restoring impact of uplift events,
we describe a teen's stressful behaviours in three groups of measures (stress intensity, posting behaviour, linguistic),
%to uniform
and model the impact of uplift events as the statistical difference between the sets of SI and U-SI in two aspects:
the two-sample based method is employed for variation detection,
and the t-test correlation is conducted to judge the monotonous correlation.

%to be add. 升华一下:意义是什么?

The rest of the paper is organized as follows.
We introduce related works in section \ref{sec:related}, 
and conduct the data observation in section \ref{sec:obs}.
The preliminaries and problem formulation are presented in section \ref{sec:problem}.
We conduct the procedure for extracting uplift events and identifying the impact interval in section \ref{sec:interval},
and introduce the detailed method for analyzing the restoring impact of uplift events in section 
\ref{sec:impact}. 
We present the experimental results in section \ref{sec:experiment}, 
and discuss the future work in section \ref{sec:conclude}.