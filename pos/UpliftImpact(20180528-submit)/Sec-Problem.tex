\section{Preliminaries}
\label{sec:problem}
Based on the observation and psychological theory,
we conduct our research under the assumption that uplift events can ease teenagers' stress,
namely the positive restoring impact of uplift events.
While stressor events stimulate human's stress,
uplift events bring positive influence and stronger restoring ability to stressed people in various situations with multi-types~\cite{Cohen1984Positive}\cite{Cohen2010Positive}\cite{Needles1990Positive}.
Taking the three stress curves in Figure~\ref{fig:example} for example,
comparing the stress curves \emph{a)}, \emph{b)} with \emph{c)},
when an uplift event (\emph{campus art festival, holiday} here) happens,
the overall stress intensity during the stressful period is reduced.
An uplift event might happen before a teen's stress caused by scheduled stressor events (\emph{example a}),
conducting lasting easing impact;
Meanwhile, an uplift event might also happen during (\emph{example b}) or at the end of the stressful period,
which might promote the teen out of current stressful status more quickly.
To study the restoring impact of an uplift event, we structure its impact from three aspects:
\begin{itemize}
\item \textbf{Impact interval of uplifts}.
To study the impact of uplift events from microblogs,
two fundamental factors are identifying the exact time when the uplift event happens,
and the corresponding stressful interval it impacts.
The temporal order between uplift events and the teen's stress series varies in different situations,
and its a challenge to match the uplift event to the right stressful interval it actually impacts.
\item \textbf{Restoring patterns of uplifts}.
As the restoring impact of uplift events relieves the teen's stress and exhibits in multiple aspects
(e.g., the changes in posting behavior, linguistic expression, and stress intensity from microblogs),
it's meaningful to extract the stress-related restoring patterns and describe the restoring impact of uplift events structurally.
\item \textbf{Quantified impact of uplifts}.
Different types of uplift events might conduct restoring impact with different intensity.
In this paper, the ultimate problem we target to solve
is how to quantify the restoring impact both qualitatively and quantitatively from teenagers microblogs.
\end{itemize}

Given an uplift event with specific type,
we consider its restoring impact by comparing the teen's behavioral measures under two situations.
As illustrated in Figure \ref{fig:SI},
in the original situation (i.e., sub-series A),
the teen's stress is caused by a stressor event, lasting for a period,
and no other intervention (namely, uplift event) occurs.
We call the set of such stressful intervals as \textbf{SI}.
In the other comparative situation (i.e., sub-series B),
the teen's stressful interval caused by the same type of stressor is impacted by an uplift event with type $x$,
we call the set of such stressful intervals as \textbf{U-SI}.
Thus the difference under the two situations SI and U-SI could be seen as the restoring impact conducted by the uplift event of type $x$.

Next, we give the formal definition for uplift events and stressor events from the perspective of linguistic structure.

\begin{figure}
\centering
\caption{Illustration of SI and U-SI stressful intervals.}
\includegraphics[width=\linewidth]{figs/UpSeries.eps}
\label{fig:SI}
\end{figure}
\begin{definition}
\textbf{Uplift event}.
Let $u$ = $[type,\{role, act,$ $descriptions\}]$ be an uplift event,
where the element \emph{role} is the subject who performs the \emph{act},
and \emph{descriptions} are the key words related to $u$.
According to psychological scales \cite{hassles,Jun2008Influence},
teenagers' uplift stressors mainly focus on six aspects,
as $\mathbb{U} =\{$ \emph{entertainment', 'school life', 'family life',
'pear relation', 'self-cognition', 'romantic'}$\}$, $\forall u$, $u._{type} \in \mathbb{U}$.
\end{definition}

\begin{definition}
\textbf{Stressor event}.
Similar to stressor event,
let $e$ = $[type,\{role, act,$ $descriptions\}]$ be a stressor event.
According to psychological questionnaires \cite{Kanner1981Comparison, scale1, scale2, scale3},
we classify stressor events into five types, as $\mathbb{S}=\{$ \emph{'school life', 'family life',
'pear relation', 'self-cognition', 'romantic'}$\}$, $\forall e$, $e._{type} \in \mathbb{S}$.
\end{definition}

The examples of teens' microblogs describing uplift events and stressor events are listed in Table \ref{tab:uplifts} and Table \ref{tab:stressors}.
For the post '\emph{I have so much homework today!!!}', its elements are \emph{role = 'I', act='have', descriptions = 'homework'}, and the \emph{type = 'school life'}.
For the post '\emph{Expecting Tomorrow' Adult Ceremony[Smile][Smile]~~}', we translate it into \emph{act = 'expecting', description = 'Adult Ceremony'}, and the \emph{type = 'self-cognition'}.

\textbf{Problem}: For an uplift event $u$ with type $U^{'}$,
a stressor event $e$ with type $S^{'}$,
let $F$:$(u, U^{'}, e, S^{'}) \rightarrow $ \bm{${A}$}
(\bm{${A}$} is a multidimensional vector) be the restoring influence of uplift event $u$
conducted on the stress caused by stressor event $e$.
We aim to quantify such influence \bm{${A}$} from multiple views.
