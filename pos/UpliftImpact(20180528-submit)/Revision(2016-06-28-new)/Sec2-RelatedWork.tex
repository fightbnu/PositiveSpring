\section{Related Work}

\subsection{Stress Detection}

\subsubsection{Stress Detection via Questionnaires}
Cohen's Perceived Stress Scale (PSS-14)~\cite{SC1983} is commonly used to measure human's stress level worldwide in psychology.
The scores of PSS-14 indicate different stress levels (i.e., none, light, moderate, and strong).
According to Ting and Huang~\cite{ZT2003}, the PSS-14 score value 26 is the boundary value of PSS-14 questionnaire. Score falling into [27,75] indicates strong stress,
score in [25,26) indicates moderate stress,
score in [1,24) indicates light stress, and score 0 indicates no stress.

Pozos-Radillo \emph{et al.}~\cite{academic} examined the correlation and predictive value between the Academic Stress Inventory (ASI) and the Stress Symptom Inventory (SSI) among 527 university students.
The findings showed that IEA situations corresponding to classroom intervention, mandatory work, and doing an exam predicted high-level chronic stress; being a female and 18, 23, and 25 years old were associated mostly to stress. It was
 concluded that accurate identification of stressors could help understand stress and its harmful effects on college students.

Michie~\cite{work} looked into pressures at work which could cause high and long lasting levels of stress, aiming to judge whether people were doing enough to prevent that harm.
Based on structural equation modeling (SEM), Wohn and LaRose~\cite{lonely} distributed and analyzed scales sampling 387 university students, to survey for life stress, learning self-efficacy, social self-efficacy, and smart phone addiction.

\subsubsection{Stress Detection via Physical and Physiological Signals}
In recent years, sensor technology is widely used for human stress status detection and monitoring, based on the changing status of various physical and psychological signals. Typically used physical signals are gestures, voice, eye gaze, facial expressions, pupil dilation, and blink rates.
In~\cite{phone1}, smart phones were used as a voice sensor by collecting voice variations when people talking naturally,
focusing on cognitive stress and stressor frequency estimation.
Using a continuous sensing App on Android phones,
Wang \emph{et al.}~\cite{Ubicomp2014} studied the impact of workload on stress, sleep,
activity, mood, sociability, mental well-being, and academic performance for college students in a class.
Kocielnik \emph{et al.}~\cite{context} presented stress information derived from sensor measurements in the context of person's activities, aiming to
provide the user with meaningful, useful, and actionable information.
Bousefsaf \emph{et al.}~\cite{mental} detected mental stress using a low-cost webcam, which could recover the instantaneous heart rate signals from video frames of human faces conveniently.

Physiological signals popularly used for stress detection include heart rate variability (HRV), electroencephalogram (EEG), electrocardiogram (ECG), galvanic skin response (GSR), blood pressures, and electromyogram~\cite{Plarre2011,Fairclough2006,Healey2005}.
Sierra \emph{et al.}~\cite{fuzzy} described the behavior of an individual under stressing stimuli in terms of HR and GSR, and implemented a fuzzy logic based stress detection system.

Combining observed driving behaviors,
Rigas \emph{et al.}~\cite{driver} integrated multiple physiological signals (i.e., skin conductance, electromyogram, electrocardiogram, and respiration) to detect drivers' stress.
Hamid \emph{et al.}~\cite{PSSEGG} proved the negative correlation between the ratio of EEG Power Spectrum and PSS,
and suggested combining physiological signals and PSS for stress detecting.
Sun \emph{et al.}~\cite{activity} collected accelerometer, ECG, and GSR signals from 20 volunteers doing activities (sitting, standing, walking),
and differed the changing of physiological signals between physical activity and mental stress.

Xu \emph{et al.}~\cite{Xu2015} proposed a cluster-based analysis method for personalized stress evaluation using physiological signals.
It exploited the homogeneity of subjects in the change of their physiological features due to stress,
and clustered subjects who shared similar patterns of stress response together. A cluster-wise stress evaluation
was then performed using the general regression neural network to accommodate inter-subject differences.

\subsubsection{Stress Detection on Social Networks}
With social network becoming a new channel for sharing personal information,
many researchers turn to detect stress by leveraging open self-expressed social media platforms (e.g., Twitter, Facebook, Tencent Weibo, and BBS).
Compared to traditional sensors, the later are low-cost and full of ``big data"~\cite{depression1,depression2}.

Shen \emph{et al.}~\cite{depression3} constructed a two-stage supervised learning framework to identify potential depression candidates,
based on the content and temporal features extracted from their write-ups on BBS.
Moreno \emph{et al.}~\cite{feelingFB} adopted negative binomial regression analysis to evaluate college students' Facebook disclosures which met DSM criteria for a depression symptom or a major depressive episode (MDE).
Campisi \emph{et al.}~\cite{Facebook} studied the relation between Facebook, stress, and incidence of URI (upper respiratory infection) in undergraduate college students, indicating that the impact of stress on the URI incidence rate increased with the size of the social network.
Jennifer \emph{et al.}~\cite{just} evaluated the disclosure of depression from college students' Facebook profiles, and proved that in-person communication from friends or trusted adults was the most preferred stress relieve method for depression detected from Facebook.
Daniel \emph{et al.}~\cite{FBbased} proved that the medical students could benefit from a stress management intervention based exclusively on Facebook through a pilot study.
Wohn and LaRose~\cite{lonely} studied the relationship between loneliness, varied dimensions of Facebook use, and college adjustment among first-year students, concluded that loneliness was a stronger indicator of college adjustment than any dimension of Facebook usage.
As the first work for impression management on Facebook through emotional disclosure,
Lin \emph{et al.}~\cite{Depression2012} showed that individuals were more likely to express positive relative to negative emotions on Facebook than in real life.
Wohn \emph{et al.}~\cite{Loneliness} investigated the relationships between loneliness, anxiousness, alcohol, and marijuana use in the prediction of college students�� connections with others on Facebook as well as their emotional connectedness to Facebook.
Xue \emph{et al.}~\cite{XueUbicomp13,HIS} investigated a number of teens typical posting behaviors that might reveal adolescent stress, and applied five classifiers to teens stress detection.
Based on the detection result, three timely intervention steps (encouraging teens to read or do something, notifying guardians at the worst case) were designed to help pressurized teenagers cope with their stress~\cite{EDBT2014}.
Lin \emph{et al.}~\cite{LinHuiJie14,LHJ} trained a deep sparse neural network to detect psychological stress from cross-media microblogs.

\subsection{Stressful Events Detection}

\subsubsection{Stressful Events Detection via Psychological Questionnaires}
Traditionally, detection of stressful events is based on psychological questionnaires for different groups of people.
Specific to adolescents, several acknowledged questionnaires and investigations are done~\cite{Journal1986,hassles,Taiwan2011,Questionaire2011,childhood,academic}.%suicide2013,moving1990,
Bobo \emph{et al.}~\cite{Journal1986} developed an Adolescent Hassles Inventory (AHI) upon the Hassles Scale~\cite{hassles}.
Using a stratified random sampling method,
Wang and Yao~\cite{Questionaire2011} developed a College Students�� Stressors Questionnaire.
Through 8 questions contained in a semi-structured journal,
Lu~\cite{Taiwan2011} identified 15 categories of everyday hassles
and 4 types of related information behaviors from 133 children in a public elementary school in an urban community in Taiwan.
%Raviv \emph{et al.}~\cite{moving1990} interviewed 73 adolescents and identified
%15 stressful elements and 19 supportive elements relating to house moving.
Li \emph{et al.}~\cite{moderate2010} adopted an organism $\times$ environment interaction approach to examine the occurrence of PIU (Problematic Internet Use)
in adolescent females and males. A mediated moderation model was tested, in which temperament moderated the relationship between
stressful life events and PIU, and this moderating effect was mediated by maladaptive cognitions about Internet use.
Besides, the sources of gender difference in PIU implied were examined in the model~\cite{moderate2010}.
The findings of the study confirmed the earlier investigation result that there was a positive relationship
between stressful life events and PIU in adolescents.
You \emph{et al.}~\cite{childhood} conducted an online survey of 5989 Chinese university students, showing that childhood adversity
and recent school life stressors were the most important predictors of suicide in this population.

An online survey made by~\cite{BevanGS14} on Facebook examined the relationship between general perceived levels of stress,
quality of life, social networking usage, and disclosing important life events on Facebook
in order to better understand
the complex relationship between online disclosure and individual well-being status in life.

Through a stress-related questionnaire for 12 weeks of 9 trial participants on Android smart phones,
Weppner \emph{et al.}~\cite{SmartPhone2013} showed that challenges were positively correlated with stress
scales, and skills were negatively correlated with stress scales.

\subsubsection{Public and Private Life Events Detection on Social Media}
Taking advantage of internet resources, public and private life events detection on social media has been studied,
utilizing burst detection, topic tracking and information diffusion techniques
%(e.g., bursty news~\cite{DBLP:conf/wsdm/SarmaJY11,DBLP:conf/naacl/PetrovicOL10},
(e.g., bursty news~\cite{DBLP:conf/wsdm/SarmaJY11},
%popular events~\cite{DBLP:conf/kdd/RitterMEC12,DBLP:conf/kdd/LinZMH10},
popular events~\cite{DBLP:conf/kdd/RitterMEC12},
scheduled city events ~\cite{DBLP:conf/acl/BensonHB11},
disasters~\cite{DBLP:conf/www/SakakiOM10},
and controversial events~\cite{DBLP:conf/cikm/PopescuP10}).

Different from public events detection, life events detection is an emerging research direction, focusing on private events related to specific users.
However, most of the work focus on text analysis of posts and ignore users' emotional behaviors along with the life events.
~\cite{Yu2011Mining}combined a supervised data mining algorithm and an unsupervised distributional semantic model to discover association language patterns from web corpus.
Chan \emph{et al.}~\cite{Event2011} applied the association rule mining technique to extract frequently appeared word pairs as features to
classify user's sentences with negative life events into predefined event categories.
Based on users' Twitter streams, Li and Cardie~\cite{DBLP:conf/www/LiC14} constructed a chronological timeline for personal important events from posts using an
unsupervised approach. Tweets talking about personal (as opposed to public) and time-specific (as opposed to time-general)
topics were included in the personal timeline.
A non-parametric multi-level Dirichlet Process model was introduced to recognize
four types of tweets: personal time-specific (PersonTS), personal
time-general (PersonTG), public time-specific (PublicTS) and public
time-general (PublicTG) topics, which, in turn, were used for further
personal event extraction and timeline generation.
Furthermore, instead of directly inspecting tweets to determine whether they correspond
to major life events, Li \emph{et al.}~\cite{DBLP:conf/emnlp/LiRCH14}
identified replies corresponding to CONGRATULATIONS or
CONDOLENCES, examined the messages they were in response to,
and then extracted event properties using Conditional Random Fields (CRF).
%Semi-supervised information harvesting techniques were used to construct one's major life event list
%\colorbox{yellow}{Li \emph{et al.}}\cite{DBLP:conf/emnlp/LiRCH14} worked under the condition that each personal event should be adequately discussed in order to be discovered.

\begin{table}
\begin{footnotesize}
\begin{center}
\begin{tabular}{|l|c|c|c|} \hline
%\begin{tabular}{|p{3.6cm}|p{1cm}|p{1cm}|p{1cm}|} \hline
\textbf{Stressor Event} & \textbf{Duration} & \textbf{Significance$^1$} & \textbf{Frequency} \\ \hline
School opens & 1 day  & 2 & 6  \\ \hline
Final/mid-term exam  & 3 days  & 2 & 30 \\ \hline
Weekly/monthly exam  &  3 days  & 2 & 34 \\ \hline
Notification of exam result &  1 day & 2 & 64 \\ \hline
Contest/competition &  1 day & 1 & 13 \\ \hline
IT exam  & 1 day  & 2 & 3 \\ \hline
%Certificate of education exam  & 3   & 2 & 3 \\ \hline
Mock exam &  3 days   & 3 & 9  \\ \hline
College entrance application & 1 day & 1 & 3   \\ \hline
College entrance exam  & 3 days  & 3 & 3 \\ \hline
College application & 3 days  & 3 & 3 \\ \hline
\end{tabular}
\begin{tablenotes}
    \item [1] Significance = 1: \emph{weak}, 2: \emph{moderate}, 3: \emph{strong}
\end{tablenotes}
\caption{Some study-related stressor events at Taicang High School from January 1, 2012 to February 1, 2015}
\label{tab:schoolEventSummary}
\end{center}
\end{footnotesize}
\end{table}


