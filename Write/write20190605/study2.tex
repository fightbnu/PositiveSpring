\section{Study2: Identify Positive Events and the Corresponding Impact Interval from microblogs}
%\subsection{Identify Uplifts and Impact Interval}
In this section,
we first introduce the procedure to extract uplift events and stressful intervals from teens' microblogs.
The uplift events are extracted from microblogs applying LTP natural language processing segmentation and parser models
\cite{che2008}.
Stressful intervals are identified using probability based statistical method according to the teen's stressful posting frequency.
We judge whether each stressful interval is correlated with neighboring uplift events,
thus to classify all stressful intervals into two sets: SI and U-SI.

\subsection{Uplift Events}
%stressor��uplift�Ķ��壬����
\emph{Linguistic structure}.
Let $u$ = $[type,\{role, act,$ $descriptions\}]$ be an uplift event,
where the element \emph{role} is the subject who performs the \emph{act},
and \emph{descriptions} are the key words related to $u$.
According to psychological scales \cite{hassles,Jun2008Influence},
teens' uplift stressors mainly focus on six aspects,
as $\mathbb{U} =\{$ \emph{entertainment', 'school life', 'family life',
'pear relation', 'self-cognition', 'romantic'}$\}$, $\forall u$, $u._{type} \in \mathbb{U}$.
Similar to uplift event,
let $e$ = $[type,\{role, act,$ $descriptions\}]$ be a stressor event.
According to psychological questionnaires \cite{scale2, scale3, Kanner1981Comparison, scale1},
we classify stressor events into five types, as $\mathbb{S}=\{$ \emph{'school life', 'family life',
'pear relation', 'self-cognition', 'romantic'}$\}$, $\forall e$, $e._{type} \in \mathbb{S}$.

\begin{table*}
\centering
\caption{\small{Examples of topic words for uplift events.}}
\label{tab:keyWords}
\small{
\begin{tabular}{lll}
\toprule
Dimension & Example words & Total \\ \midrule
\emph{entertainment}  & hike, travel, celebrate, dance, swimming, ticket, shopping, air ticket, theatre, party, Karaoke,& 452\\
                      & self-driving tour, game, idol, concert, movie, show, opera, baseball, running, fitness, exercise & \\
\emph{school life}    & reward, come on, progress, scholarship,admission, winner, diligent, first place, superior & 273\\
				      & hardworking, full mark,  praise, goal, courage, progress, advance, honor, collective honor& \\
\emph{romantic}       &  beloved, favor, guard, anniversary,  concern, tender, deep feeling, care, true love, promise, & 138\\
				      & cherish, kiss, embrace, dating, reluctant, honey, sweetheart, swear, love, everlasting, goddess &\\
\emph{pear relation}  & listener, company, pour out, make friends with, friendship, intimate, partner, team-mate, brotherhood& 91\\
\emph{self-cognition} & realize, achieve, applause, fight, exceed, faith, confidence, belief, positive, active, purposeful & 299\\
\emph{family life}    & harmony, filial, reunite, expecting, responsible, longevity, affable, amiability, family, duty & 184\\
\bottomrule
\end{tabular}}
\end{table*}

\emph{Lexicon}.
We construct our lexicon for six-dimensional uplift events from two sources.
The basic positive affect words are selected from the psychological lexicon SC-LIWC (e.g., \emph{expectation}, \emph{joy}, \emph{love} and \emph{surprise})\cite{Tausczik2010The}.
Then we build six uplift event related lexicons by expanding the basic positive words from the data set of teens' microblogs,
and divide all candidate words into six dimensions corresponding to six types of uplift events,
containing 452 phrases in \emph{entertainment},
184 phrases in \emph{family life},
91 phrases in \emph{friends},
138 phrases in \emph{romantic},
299 phrases in \emph{self-recognition} and 273 phrases in \emph{school life}, with totally 2,606 words,
as shown in Table \ref{tab:keyWords}.
Additionally, we label \emph{role} words (i.e., \emph{teacher}, \emph{mother}, \emph{I, we}) in the uplift lexicon.

%\begin{table}[H]
%\begin{center}
%\caption{\small{Structured extraction of stressor events from microblogs.}}
%\small{
%\begin{tabular}{l} \hline \rowcolor{gray!40}
%I don't know how long can I bear the nag.\\ \rowcolor{gray!40}
%(Doer:\emph{I}, Act:\emph{bear}, Object:\emph{nag})\\ %\hline
%Parents like to judge everything around me with their emotion.
%\\(Doer:\emph{parents}, Act:\emph{judge}, Object:\emph{everything})\\ \rowcolor{gray!40}
%%Hope that my mother could revive earlier.\\ \rowcolor{gray!40}
%%(Doer:\emph{my mother}, Act:\emph{revive})\\%\hline
%Every one betrayed me. \\ \rowcolor{gray!40}
%(Doer:\emph{every one}, Act:\emph{betray}, Object:\emph{me})\\ \hline
%I'm too weak to handle such a fierce competition.\\ %\rowcolor{gray!40}
%(Doer:\emph{I}, Act:\emph{too weak to handle}, Object:\emph{competition})\\ \rowcolor{gray!40}
%%I just felt hurt, depressed, self-abased and sad.
%%\\(Doer:\emph{I}, Act:\emph{feel hurt, depressed, self-abased and sad})\\ \rowcolor{gray!40}%\hline
%%My holiday is filled with all kinds of homework.\\ \rowcolor{gray!40}
%%(Doer:\emph{My holiday}, Act:\emph{fill with}, Object:\emph{homework})\\ %\hline
%%Unescapably, it's time to go back to school.
%%\\(Act:\emph{go back}, Object:\emph{school})\\ \rowcolor{gray!40} %\hline
%When can you be aware of my heart-broken feelings?\\\rowcolor{gray!40}
%(Doer:\emph{you}, Act:\emph{be aware of}, Object:\emph{heart-broken feelings})\\ \hline
%\end{tabular}
%}
%\label{tab:stressors}
%\end{center}
%\end{table}

\emph{Parser relationship}.
For each post, after word segmentation, we parser current sentence to find its linguistic structure,
and then match the main linguistic components with uplift event related lexicons in each dimension.
The parser model in Chinese natural language processing platform \cite{Che2010, che2008} is adopted in this part,
which identifies the central verb of current sentence first, namely the \emph{act},
and constructs the relationship between the central verb and corresponding \emph{role} and \emph{objects} components.
By searching these main elements in uplift event related lexicons,
we identify the existence and type of any uplift event.
Due to the sparsity of posts, the \emph{act} might be empty.
The \emph{descriptions} are collected by searching all nouns, adjectives and adverbs in current post.
In such way, we extract structured uplift events from teens' microblogs.

\subsection{Impact Interval of Current Positive Event}
Basically, in this part, we identify stressful intervals from time line thus to support further quantifying the influence of an uplift event.
Splitting interval is a common time series problem, and various solutions could be referred.
Here we identify the teen's stressful intervals in three steps.

In the first step, we extract uplift events, stressor events and filter out candidate intervals after a smoothing process.
Since a teen's stress series detected from microblogs are discrete points,
the loess method \cite{Cleveland1988Locally} is adopted to highlight characteristics of the stress curve.
The settings of parameter \emph{span} will be discussed in the experiment section,
which represents the percentage of the selected data points in the whole data set
and determines the degree of smoothing.
The details are present as Algorithm \ref{alg:alg1} of the appendix.

In the second step, applying the Poisson based statistical method proposed in~\cite{Li2017Analyzing},
we judge whether each candidate interval is a confidential stressful interval.
The details are present as Algorithm \ref{alg:alg2} of the appendix.

Finally, we divide the stressful intervals into two sets: the SI set and the U-SI set,
according to its temporal order with neighboring uplift events.
The details are present as Algorithm \ref{alg:alg3} of the appendix.

\subsection{Results}
%example
The examples of teens' microblogs describing uplift events are listed in Table \ref{tab:uplifts}. % and Table \ref{tab:stressors}.
%For the post '\emph{I have so much homework today!!!}', its elements are \emph{role = 'I', act='have', descriptions = 'homework'}, and the \emph{type = 'school life'}.
For the post '\emph{Expecting Tomorrow' Adult Ceremony[Smile][Smile]~~}',
we translate it into \emph{act = 'expecting', object = 'Adult Ceremony'},
and \emph{type = 'self-cognition'}. 

To check the performance of uplift event extraction and the validation of our assumption,
we first identify uplift events and corresponding restoring performance from microblogs,
and compare the results with scheduled positive events collected from the school's official web site.

\begin{table}[H]
\begin{center}
\caption{\small{Structured extraction of positive events from microblogs.}}
\small{
\begin{tabular}{l} \hline \rowcolor{gray!40}
I am really looking forward to the spring outing on Sunday now. \\ \rowcolor{gray!40}
(Doer:\emph{I}, Act:\emph{looking forward}, Object:\emph{spring outing})\\
My holiday is finally coming [smile]. \\
(Doer:\emph{My holiday}, Act:\emph{coming}, Object:\emph{[smile]})\\ \rowcolor{gray!40}%\hline
First place in my lovely math exam!!! In memory of it.\\ \rowcolor{gray!40}
Object:\emph{first place, math, exam, memory})\\ %\hline
You are always here for me like sunshine. \\
(Doer:\emph{You}, Object:\emph{sunshine})\\ \rowcolor{gray!40} %\hline
Thanks all my dear friends taking the party for me. \\ \rowcolor{gray!40}
Happiest birthday!!!\\ \rowcolor{gray!40}
(Doer:\emph{friends}, Act:\emph{thanks}, Object:\emph{party, birthday})\\
%Be yourself. Trust yourself and follow your heart. \\
%(Doer:\emph{yourself}, Act:\emph{trust}, Object:\emph{heart})\\ \rowcolor{gray!40} %\hline
%Feel proud of our play in the Games. Our class is always the family!!!\\ \rowcolor{gray!40}
%(Doer:\emph{Our}, Object:\emph{class, family})\\
%A good film always makes bring comfort and happiness to me.\\
%(Doer:\emph{me}, Act:\emph{bring}, Object:\emph{comfort, happiness})\\ \rowcolor{gray!40}%\hline
I know my mom is the one who support me forever, no matter \\ %\rowcolor{gray!40}
when and where. (Doer:\emph{mom}, Act:\emph{support})\\ \rowcolor{gray!40}
Expecting Tomorrow' Adult Ceremony[Smile][Smile]~~\\ \rowcolor{gray!40}
(act: \emph{expecting}, object:\emph{Adult Ceremony})\\ \hline
\end{tabular}}
\label{tab:uplifts}
\end{center}
\end{table}
