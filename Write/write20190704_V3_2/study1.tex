\section{Pilot study: Observation on the stress-buffering function of school scheduled positive events}
\label{sec:study1}
\subsection{Participants}
We built our dataset based on two sources: 1) the microblogs of students coming from Taicang High School,
collected from January 1st, 2012 to February 1st, 2015;
and 2) list of scheduled school events, with exact start and end time.
We filtered out 124 active students according to their posting frequency from over 500 students,
and collected their microblogs throughout the whole high school career. Totally 29,232 microblogs are collected in this research,
where 236 microblogs per student on average, 1,387 microblogs maximally and 104 posts minimally. 

\subsection{Measures}
\emph{School-scheduled positive events}.
The list of weekly scheduled school events (from February 1st, 2012 to August 1st 2017) are collected from the school's official website
\footnote{http://stg.tcedu.com.cn/col/col82722/index.html}, with detailed event description and grade involved in the event.
There are 122 stressor events and 75 uplift events in total.
Here we give the examples of scheduled uplift and stressor events in high school life, as shown in Table~\ref{tab:example}.
Comparing the stress curves \emph{a)}, \emph{b)} with \emph{c)},
when an uplift event (\emph{campus art festival, holiday} here) happens,
the overall stress intensity during the stressful period is reduced.
An uplift event might happen before a teen's stress caused by scheduled stressor events (\emph{example a}),
conducting lasting easing impact;
Meanwhile, an uplift event might also happen during (\emph{example b}) or at the end of the stressful period,
which might promote the teen out of current stressful status more quickly.
There are 2-3 stressor events and 1-2 positive event scheduled per month in current study.

\begin{table}[H]
\caption{\small{Examples of school scheduled positive and stressor events.}}
\label{tab:example}
\resizebox{.45\textwidth}{9mm}{
\small{
\begin{tabular}{cccc}
\toprule
Type & Date	& Content	& Grade	\\
\midrule
\emph{stressor event} & 2014/4/16 & \emph{first day of mid-term exam} & grade1,2\\
\emph{uplift event} & 2014/11/5 & \emph{campus art festival} & grade1,2,3\\
\bottomrule
\end{tabular}
}
}
\end{table}

\emph{Stress detected from microblogs}.
Since our target is to observe the restoring impact of positive events for teenagers under stress,
based on previous research~\cite{XueUbicomp13},
we detected the stress level (ranging from 0 to 5) for each post;
and for each student, we aggregated the stress during each day by calculating the average stress of all posts.
To protect the privacy, all usernames are anonymized during the experiment
The positive level (0-5) of each post is identified based on the frequency of positive words (see Section 5 for details).
Figure~\ref{fig:example} shows three examples of a student's stress fluctuation during three mid-term exams,
where the positive event \emph{campus art festival} was scheduled ahead of the first exam,
the positive event \emph{holiday} happened after the second exam,
and no scheduled positive event was found nearby the third exam.
The current student exhibited differently in above three situations, with the stress lasting for different length and with different intensity.
\begin{figure}
\centering
\caption{Examples of school related stressor events, uplift events and a student's stress fluctuation}
\includegraphics[width=\linewidth]{figs/exampleWave.eps}
\label{fig:example}
\end{figure}

\subsection{Method}
To further observe the influence of positive events for students facing stressor events,
we statistic all the stressful intervals~\cite{Li2017Analyzing} detected surround the scheduled examinations over the 124 students during their high school career.
For each student, we divide all the stressful intervals into two sets:
1) In the original sets, stress is caused by a stressor event, lasting for a period,
and no other intervention (namely, uplift event) occurs.
We call the set of such stressful intervals as \textbf{SI};
2) In the other comparative sets,
the teen's stressful interval is impacted by a positive event $x$,
we call the set of such stressful intervals as \textbf{U-SI}.
Thus the difference under the two situations could be seen as the restoring impact conducted by the positive event of type $x$.
Based on the scheduled time of stressor and positive events,
we identified 518 scheduled academic related stressful intervals (SI)
and 259 academic stressful intervals impacted by four typical scheduled positive events (U-SI) (in Table \ref{tab:schedule})
from the students' microblogs.
Further observations are conducted on the two sets to verify the impact of positive events from multi perspectives.

\subsection{Results}
Figure~\ref{fig:frequency} shows five measures of each teen during the above two conditions:
the \emph{accumulated stress}, the \emph{average stress} (per day), the \emph{length of stressful intervals},
the \emph{frequency of academic topic words}, and the \emph{ratio of academic stress among all types of stress}.
For each measure, we calculate the average value over all eligible slides for each student.
Comparing each measure in scheduled exam slides under the two situations:
1) existing neighbouring positive events or 2) no neighbouring scheduled positive events,
we find that students during exams with neighbouring positive events exhibit less average stress intensity
(both on accumulated stress and average stress),
and the length of stress slides are relatively shorter.
%H1: stress length��intensity�½�

Further, we statistic the frequency of academic related topic words for each exam slide
(as listed in Table \ref{tab:studyWords}),
and look into the ratio of academic stress among all five types of stress.
Results in Figure~\ref{fig:frequency} shows that most students talked less about the upcoming or just-finished exams when positive events happened nearby,
with lower frequency and lower ratio.

\begin{table}[h]
\centering
\caption{\small{Examples of academic topic words from microblogs.}}
\label{tab:studyWords}
\small{
\begin{tabular}{c}
\toprule
exam, fail, review, score, test paper, rank, pass, math, chemistry\\
homework, regress, fall behind, tension, stressed out, physics,\\
nervous, mistake, question, puzzle, difficult, lesson, careless\\
\bottomrule
\end{tabular}
}
\end{table}

The statistic result shows clues about the stress-buffering function of scheduled positive events,
which are constant with the psychological theory ~\citep{Cohen1984Positive, Cohen2010Positive, Needles1990Positive}, 
indicating the reliability and feasibility of the microblog data set. 
However, 
this is an observation based on specific scheduled events, 
and cannot satisfy our need for automatic, timely, and continuous perception of stress-buffering. 
Therefore, in study 1, we will propose a framework to automatically detect positive events and its impact interval.
Based on this, in study 2, we will examine whether the stress-buffering function of the automatically extracted positive events is related to the microblogging measures (posting behavior, stress intensity, linguistic expressions), and explore its function mode.
 