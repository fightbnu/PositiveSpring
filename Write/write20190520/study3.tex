\section{Study3: Identify Uplifts and Impact Interval from microblogs}
%\subsection{Identify Uplifts and Impact Interval}
In this section,
we first introduce the procedure to extract uplift events and stressful intervals from teens' microblogs.
The uplift events are extracted from microblogs applying LTP natural language processing segmentation and parser models
\cite{che2008}.
Stressful intervals are identified using probability based statistical method according to the teen's stressful posting frequency.
We judge whether each stressful interval is correlated with neighboring uplift events,
thus to classify all stressful intervals into two sets: SI and U-SI.

\subsection{Uplift Events}
\emph{Lexicon}.
We construct our lexicon for six-dimensional uplift events from two sources.
The basic positive affect words are selected from the psychological lexicon SC-LIWC (e.g., \emph{expectation}, \emph{joy}, \emph{love} and \emph{surprise})\cite{Tausczik2010The}.
Then we build six uplift event related lexicons by expanding the basic positive words from the data set of teens' microblogs,
and divide all candidate words into six dimensions corresponding to six types of uplift events,
containing 452 phrases in \emph{entertainment},
184 phrases in \emph{family life},
91 phrases in \emph{friends},
138 phrases in \emph{romantic},
299 phrases in \emph{self-recognition} and 273 phrases in \emph{school life}, with totally 2,606 words,
as shown in Table \ref{tab:keyWords}.
Additionally, we label \emph{role} words (i.e., \emph{teacher}, \emph{mother}, \emph{I, we}) in the uplift lexicon.

\emph{Parser relationship}.
For each post, after word segmentation, we parser current sentence to find its linguistic structure,
and then match the main linguistic components with uplift event related lexicons in each dimension.
The parser model in Chinese natural language processing platform \cite{Che2010, che2008} is adopted in this part,
which identifies the central verb of current sentence first, namely the \emph{act},
and constructs the relationship between the central verb and corresponding \emph{role} and \emph{objects} components.
By searching these main elements in uplift event related lexicons,
we identify the existence and type of any uplift event.
Due to the sparsity of posts, the \emph{act} might be empty.
The \emph{descriptions} are collected by searching all nouns, adjectives and adverbs in current post.
In such way, we extract structured uplift events from teens' microblogs.

\begin{table}
\centering
\caption{Examples of topic words for uplift events.}
\label{tab:keyWords}
\begin{tabular}{lll}
\toprule
Dimension & Example words & Total \\ \midrule
\emph{entertainment}  & hike, travel, celebrate, dance, swimming, ticket, & 452\\
& shopping, air ticket, theatre, party, Karaoke, & \\
& self-driving tour, game, idol, concert, movie, & \\
\emph{school life}       & reward, come on, progress, scholarship, & 273\\
				  & admission, winner, diligent, first place, superior, & \\
				  & hardworking, full mark,  praise, goal, courage, & \\
\emph{romantic}         &  beloved, favor, guard, anniversary,  concern,  & 138\\
				  & tender, deep feeling, care, true love, promise, & \\
				  & cherish, baby, kiss, embrace, dating, reluctant,&\\
\emph{pear relation}   & listener, company, pour out, make friends with, & 91\\
				 & friendship, intimate, partner, team-mate,&\\
\emph{self-cognition} & realize, achieve, applause, fight, exceed, faith, & 299\\
				 & confidence, belief, positive, active, purposeful, &\\
\emph{family life}       & harmony, filial, reunite, expecting, responsible, & 184\\
				& longevity, affable, amiability, family, duty, &\\
\bottomrule
\end{tabular}
\end{table}

\subsection{Impact Interval}
When an uplift event happens, it can conduct positive impact on stressed teens,
which may exhibit in multi-perspectives,
including the length of corresponding stressful interval, the stress intensity and stressful behaviors during the interval.
Basically, in this part, we identify stressful intervals from timeline thus to support further quantifying the influence of an uplift event.
Splitting interval is a common time series problem, and various solutions could be referred.
Here we identify the teen's stressful intervals in three steps.
In the first step, we extract uplift events, stressor events and filter out candidate intervals after a smoothing process.
Then, applying the Poisson based statistical method proposed in~\cite{Li2017Analyzing},
we judge whether each candidate interval is a confidential stressful interval.
Finally, we divide the stressful intervals into two sets: the SI set and the U-SI set,
according to its temporal order with neighboring uplift events.

\emph{Smooth stress series}.
Since a teen's stress series $S=\{s_1, s_2, \cdots, s_n\}$ detected from microblogs (aggrated by day) are discrete points,
we adopt the loess (local regression using weighted linear least squares and a $2^{nd}$ degree polynomial) method \cite{Cleveland1988Locally} to highlight characteristics of the stress curve.
In loess method, we need to set the parameter \emph{span},
which represents the percentage of the selected data points in the whole data set (ranging from 0 to 1),
and determines the degree of smoothing.
We discuss the influence of different \emph{span} settings on the result of interval segmentation in the experiment section.
%to be add 1: interval parameter.

\emph{Select candidate intervals.}
We filter out candidate stressful intervals on the smoothed stress series $S^{'}=\{s^{'}_1,s^{'}_2,\cdots,s^{'}_n\}$.
%As illustrated in Figure xx, w
Let the sub-series $w_{<a,b>} = [s^{'}_a, \cdots, s^{'}_b]$ as a \emph{wave},
where $s^{'}_v$ $= {vally(w_{<a,b>})}$ is the minimum stress value,
$s^{'}_p$ $= peak(w_{<a,b>})$ is the maximal stress value during $\{s^{'}_a,\cdots,s^{'}_b\}$,
and $s^{'}_a \leq s^{'}_{a+1} \leq \cdots \leq s^{'}_p \leq s^{'}_{p+1} \leq \cdots \leq s^{'}_b$.
A candidate interval $I = <w_1,\cdots, w_i,\cdots, w_m>$ is identified with following rules:
\begin{itemize}
\item $s^{'}_1 = 0$, $s^{'}_m = 0$. $\forall s^{'}_j \in \{s^{'}_2,\cdots,s^{'}_{m-1}\}$, $s^{'}_j > 0$.
\item Let wave $w_i$ be the biggest wave in current candidate interval, with $peak(w_i) = \omega$,
$\forall $ wave $w_j \in I$, $peak(w_j)<=peak(w_i)$.
\item For waves $w_k$ before the interval biggest wave $w_i$ , i.e., $\forall w_k \in <w_1,\cdots,w_{i-1}>$,
$peak(w_{k+1})>=peak(w_k)$, $vally(w_{k+1}) >= peak(w_k)$.
\item For waves $w_k$ behind the interval biggest wave $w_i$, i.e.,  $w_k \in <w_{i}, \cdots, w_m>$,
$peak(w_{k+1})<=peak(w_k)$, $vally(w_{k+1}) <= peak(w_k)$.
\end{itemize}

\emph{Identify stressful intervals.}
For each candidate interval,
a Poisson based probability model~\cite{Li2017Analyzing} is adopted to measure how confidently the current interval is a stressful interval.
Here a teen's stressful posting rate under stress ($\lambda_1$) and normal conditions ($\lambda_0$) are modeled as two independent poisson process:
\begin{equation}
Pr[N=n|\lambda_i]=\frac{e^{-\lambda_i T}{(\lambda_i T)}^n}{n!}
\end{equation}
where $i\in\{0,1\}$, $n=0,1,\cdots,\infty$.
We expect that $\lambda_1 > \lambda_0$, and measure the probability as $P(\lambda_1>\lambda_0|N_1, T_1, N_0, T_0)$,
where $N_1, N_0$ are the number of stressful posts, and $T_1, T_0$ are time duration corresponding to $\lambda_1$ and $\lambda_0$.
Without loss of generality, we assume a Jeffreys non-informative prior on $\lambda_1$ and $\lambda_0$,
and infer the posterior distribution $P(\lambda_1|N_1)$ and $P(\lambda_0|N_0)$ according to Bayes Rule.
Thus for current interval $I_1$ and historical normal interval $I_0$,
the quantified probability $\beta = P(\lambda_1>\lambda_0|I_1,I_0)$ $\in (0,1)$ indicates the confidence whether $I_1$ is a stressful interval.

\emph{Intervals impacted by uplift events.}
In this part, we filter out two sets of stressful intervals: stressful intervals without the impact of uplift events (SI),
and stressful intervals under the impact of uplift events (U-SI).
For a detected stressful interval $I = <t_1,\cdots,t_n>$, we consider the temporal order between $I$ and any detected uplift event $u$ happened at time point $t_u$:
\begin{itemize}
\item If the uplift event $u$ happens during the stressful interval, i.e., $t_u \in [t_1,t_n]$, the uplift interval $I$ is judged as $I \in SI$.
\item For the uplift event happening nearby a stressful interval,
we also consider the probability that it conducts impact on the teen's stressful interval.
Here the gap between $t_u$ and $I$ is limited to $\xi$, i.e.,
if $t_u \in [t_{1}-\xi, t_1)\cup(t_{n},t_{n}+\xi]$, then $I \in SI$.
%The setting of parameter $\xi$ is discussed in experiment section.
\end{itemize}
If a stressful interval satisfies none of the above conditions, we classify it into the U-SI set.