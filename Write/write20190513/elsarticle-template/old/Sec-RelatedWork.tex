\section{Related Work}
\label{sec:related}
\subsection{Protective function of uplift events}
Many psychological researchers have focused on the restorative function of positive events and emotions with respect to physiological, psychological, and social coping resources.
Folkman \emph{et al.}\cite{Folkman2010Stress} identified three classes of coping mechanisms that are associated with positive emotion during chronic stress: positive reappraisal, problem-focused coping, and the creation of positive events.
The author also considered the possible roles of positive emotions in the stress process, and incorporated positive emotion into a revision of stress and coping theory in
the work \cite{Folkman1997Positive}.
They conducted a longitudinal study of the care giving partners of men with AIDS and described coping processes that were associated with positive psychological states in the context of intense distress.
Cohen \emph{et al.} \cite{Cohen2010Positive} hypothesized that protective effect of uplift events operates in both directly (i.e., more positive uplift events people experienced, the less distress they experience) and indirectly ways by 'buffering'.

Chang \emph{et al.} \cite{Chang2015Loneliness} investigated the protective effect of positive events in a sample of 327 adults, and found that the positive association between loneliness and psychological maladjustment was found to be weaker for those who experienced a high number of positive life events, as opposed to those who experienced a low number of positive life events.
This is assistant with the conclusion made by Kleiman \emph{et al.}
\cite{Evan2014Social} that positive events act as protective factors against suicide individually and synergistically when they co-occur, by buffering the link between important individual differences risk variables and maladjustment.
Through exploring naturally occurring daily stressors, Ong \emph{et al.}
\cite{Ong2006Psychological} found that over time,
the experience of positive emotions functions to assist high-resilient individuals to recover effectively from daily stress.
In the survey made by Santos \emph{et al.} \cite{Santos2013The}, strategies of positive psychology are checked as potentially tools for the prophylaxis and treatment of depression, helping to reduce symptoms and for prevention of relapses.
Through a three-week longitudinal study, Bono \emph{et al.}
\cite{Bono2013Building} examined the correlation between employee stress and health and positive life events, and concluded that naturally occurring positive events are correlated with decreased stress and improved health.

\subsection{Measuring the Impact of Uplift Events}
To measure the impact of uplift events,
Doyle \emph{et al.} \cite{Kanner1981Comparison} conducted \emph{Hassles and Uplifts Scales},
and concluded that the assessment of daily hassles and uplifts might be a better approach to the prediction of adaptational outcomes than the usual life events approach.
Silva \emph{et al.} \cite{Silva2008The} presented the \emph{Hassles \& Uplifts Scale} to assess the reaction to minor every-day events in order to detect subtle mood swings and predict psychological symptoms.
To measure negative interpretations of positive social events,
Alden \emph{et al.} \cite{Alden2008Social} proposed the interpretation of positive events scale (\emph{IPES}), and analyzed the relationship between social interaction anxiety and the tendency to interpret positive social events in a threat-maintaining manner.
Mcmillen \emph{et al.} \cite{Mcmillen1998The} proposed the \emph{Perceived Benefit Scales} as the new measures of self-reported positive life changes after traumatic stressors, including lifestyle changes, material gain, increases in selfefficacy, family closeness, community closeness, faith in people, compassion, and spirituality.
Specific for college students,
Jun-Sheng \emph{et al.} \cite{Jun2008Influence} investigated in 282 college students using the \emph{Adolescent Self-Rating Life Events Checklist}, and found that the training of positive coping style is of great benefit to improve the mental health of students.

\subsection{Analyzing adolescent stress from social media}
With the high development of social network,
researchers tend to digging user' psychological status from the self-expressed public data source.
Billions of people record their life, share multi-media content, and communicate with friends through such platforms, e.g.,
Tencent Microblog, Twitter, Facebook and so on.
Inspired by rich microblogging content,
Xue \emph{et al.} \cite{XueUbicomp13, Xue2014Detecting} proposed to detect adolescent stress from single microblog utilizing machine learning methods by extracting stressful topic words, abnormal posting time, and interactions with friends.
Lin \emph{et al.} \cite{Lin2014User} construct a deep neural network to combine the high-dimensional picture semantic information into stress detecting.
Based on the stress detecting result,
Li \emph{et al.} \cite{Li2015Predicting}\cite{Li2015Using}\cite{Li2015When} adopted a series of multi-variant time series prediction techniques (i.e., Candlestick Charts, fuzzy Candlestick line and  SVARIMA model) to predict the future stress trend and wave.
Taking the linguistic information into consideration,
Li \emph{et al.} \cite{Li2017Exploring} employed a NARX neural network to predict a teen's future stress level referred to the impact of co-experiencing stressor events of similar companions.
All above pioneer work focused on the generation and development of teens' stress, providing solid basic techniques for broader stress-motivated research from social networks.

To find the source of teens' stress, previous work \cite{Li2017Analyzing} developed a frame work to extract stressor events from microblogging content and filter out stressful intervals based on teens' stressful posting rate.
Based on such research background, this paper starts from a completely new perspective, and focuses on the buffering effect of positive events on restoring stress.
Thus we push forward the study from how to find stress to the next more meaningful stage: how to deal with stress.

\subsection{Correlation analysis for multivariate time series}
Basic correlation analysis methods on time series focused on univariate data have been well studied.
As the most widely adopted method,
the Pearson correlation analysis \cite{Cohen1988Statistical} measures the linear correlation between two variables $X$ and $Y$.
One inevitable defect is that Pearson correlation is too sensitive to outlier values.
To overcome such drawback,
Spearman Rank correlation \cite{C1987The}
and Kendall Rank correlation \cite{Mcleod2011Kendall}
are proposed based on Pearson correlation.
While Pearson correlation estimates linear relationships,
Spearman correlation estimates monotonic relationships (whether linear or not),
and are calculated as the Pearson correlation between the rank values of two variables.
The Kendall correlation mainly assesses the similarity of the orderings of the data when ranked by each of the quantities.
The above correlation methods are usually used to estimate relationship between single-dimensional variables,
and cannot be adopted directly in our microblog content based scenario.

For multivariate time series analysis, two-sample based methods are widely adopted.
Such kind of methods are deduced to check whether two samples come from the same underlying distribution, which is assumed to be statistically unknown.
Correspondingly, various kernel
\cite{Sch2006A} and distance-based methods \cite{Schilling1986Multivariate}
(e.g., the nearest neighbor based method two-sample method) are proposed.
Scholkopf \emph{et al.} \cite{Sch2006A} proposed to transform the distance between two variables and nearest neighbors into a reproducing kernel Hilbert space (RKHS), and solve the problem using Maximum Mean Discrepancy.
In work \cite{Schilling1986Multivariate},
Schilling \emph{et al.} adopted the $r$-nearest neighbor based method to partition two set of event driven time series data.
The global proportion of the right divided neighbors are calculated to estimate whether there exists statistically difference between the two sets.
We use the $r$-nearest neighbor based two-sample method in our problem, thus to measure the distance and correlation between two multi-dimension variables.
