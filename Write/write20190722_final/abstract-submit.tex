Stress is viewed as the leading cause of public mental health issues.
Positive events, however, could act as a buffer against stress.
Since the stress-buffering effect of positive events in previous studies was mainly examined
by subjective self-reporting,
continuous tracking research at individual behavioral levels still remains to be explored.
In this study,
We collected microblogs (n=29,232) from a high school student group (n=500) to examine the relationship
between positive events and stress-buffering pattern based on microblog content and behavioral characteristics.
Through a pilot study we found that the stress-buffering pattern of school scheduled positive events (n=259)
was manifested in both the reduction of stress intensity,
the shorter duration of stress intervals,
and talking less about academic words on micro-blog.
Hypothetical tests for stress-buffering pattern and monotonic effect of stress changes
were further conducted based on automatical extracted positive events (n=1,914) from microblogs.
The stress-buffering pattern of positive events
was closely correlated with posting behavior (ratio = 80.65%, SD=1.96),
stress change mode (ratio = 67.74%, SD=2.04) and microblog linguistic expressions (ratio = 74.19%, SD=2.07).
Positive events conducted most intensive stress-buffering impact on 'family life' (ratio = 83.87%, SD=2.72),
followed by 'peer relationships' (ratio = 71.77%, SD=4.04) and 'school life' (ratio = 67.74%, SD=2.71) dimensions.
Positive events buffered monotonous stress changes at both the early (11.88% reduction) and late stages (5.88% reduction).
This study could inform the use of social network to reach and track the mental status transition of adolescents under stress.
The theoretical and practical implications, limitations of this study and future work were discussed.