\section{Study2: relationship between stress-buffering effect of automatically extracted positive events and microblog characteristics}

\subsection{Positive events automatically extracted from microblogs}
Since events in study 1 are scheduled and limited,
in this part we first introduce the procedure to extract positive event and its intervals from microblogs,
thus to extend our study to various types of positive events expressed in microblogs.
%Our automatically extraction accuracy was verified
%by comparing extracted positive events with scheduled school events in coincident time intervals.

\paragraph{Linguistic structure}
Let $u$ = $[type,\{doer, act,$ $description\}]$ be a positive event,
where the element \emph{doer} is the subject who performs the \emph{act},
and \emph{descriptions} are the key words related to $u$.
According to psychological scales \citep{hassles,Jun2008Influence},
adolescents' positive events mainly focus on six aspects,
as $\mathbb{U} =\{$ 'entertainment', 'school life', 'romantic', 'pear relationship', 'self-cognition', 'family life'$\}$.
% $\forall u$, $u._{type} \in \mathbb{U}$.
%Similar to positive event,
%let $e$ = $[type,\{role, act,$ $descriptions\}]$ be a stressor event.
%According to psychological questionnaires \cite{scale2, scale3, Kanner1981Comparison, scale1},
%we classify stressor events into five types, as $\mathbb{S}=\{$ \emph{'school life', 'family life',
%'pear relation', 'self-cognition', 'romantic'}$\}$, $\forall e$, $e._{type} \in \mathbb{S}$.

\paragraph{Lexicon}
We constructed our lexicon for six-dimensional positive events from two sources.
The basic positive words are selected from the psychological lexicon SC-LIWC (e.g., \emph{expectation}, \emph{joy}, \emph{love} and \emph{surprise})~\citep{Tausczik2010The}.
Then we built six topic lexicons by expanding basic positive words from adolescent microblogs,
containing 452 phrases in 'entertainment',
273 phrases in 'school life',
138 phrases in 'romantic',
91 phrases in 'peer relationship',
299 phrases in 'self-recognition' and 184 phrases in 'family life', with totally 2,606 phrases,
as examples shown in table \ref{tab:keyWords}.
Additionally, we labeled \emph{doer} words (i.e., \emph{teacher}, \emph{mother}, \emph{I, we}) in the positive lexicon.

\paragraph{Parser relationship}
For each post, after word segmentation, we parsed current sentence to find its linguistic structure,
and then matched the main linguistic components with positive topic lexicon in each dimension.
The parser model in Chinese natural language processing platform \citep{Che2010} was adopted in this part,
which identified the central verb of current sentence first, namely the \emph{act},
and constructed the relationship between the central verb and corresponding \emph{doer} and \emph{description} components.
By searching these main elements in positive event related lexicons,
we identified the existence and type of positive events.
Due to the sparsity of posts, \emph{act} might be empty.
\emph{Descriptions} were collected by searching all nouns, adjectives and adverbs.
In such way, we extracted structured positive events from microblogs.

Examples of adolescents' microblogs describing positive events are listed in table \ref{tab:uplifts}.
For the post 'Thanks all my dear friends hosting the party. Happiest birthday!!!',
we translated it into \emph{doer='friends', act = 'expecting', description = 'party'},
and \emph{type = 'entertainment'}.
To check the accuracy of positive event extraction,
in study 3,
we identified positive events and its corresponding stress-buffering effect from microblogs,
and compared the results with positive events in school planning.

\begin{table}
\begin{center}
\caption{\small{Structured extraction of positive events from microblogs.}}
\small{
\begin{tabular}{l} \hline \rowcolor{gray!40}
I am really looking forward to the spring outing on Sunday now. \\ \rowcolor{gray!40}
(doer:\emph{I}, act:\emph{looking forward}, description:\emph{spring outing})\\
My holiday is finally coming [smile]. \\
(doer:\emph{My holiday}, act:\emph{coming}, description:\emph{[smile]})\\ \rowcolor{gray!40}%\hline
First place in my lovely math exam!!! In memory of it.\\ \rowcolor{gray!40}
description:\emph{first place, math, exam, memory})\\ %\hline
You are always here for me like sunshine. \\
(doer:\emph{You}, description:\emph{sunshine})\\ \rowcolor{gray!40} %\hline
Thanks all my dear friends hosting the party.
Happiest birthday!!!\\ \rowcolor{gray!40}
(doer:\emph{friends}, act:\emph{thanks}, description:\emph{party, birthday})\\
I know my mom is the one who support me forever, no matter \\
when and where. (doer:\emph{mom}, act:\emph{support})\\ \rowcolor{gray!40}
Expecting Tomorrow' Adult Ceremony[Smile][Smile]~~\\ \rowcolor{gray!40}
(act: \emph{expecting}, description:\emph{Adult Ceremony})\\ \hline
\end{tabular}}
\label{tab:uplifts}
\end{center}
\end{table}

\paragraph{Impact Interval of Positive Event}
Next, we identified the impact interval of each positive event thus to further study its stress-buffering pattern.
Splitting interval is a common time series problem, and here we identified the target interval in three steps.
In the first step, we extracted positive events, stressor events~\citep{Li2017Analyzing} and filtered out candidate intervals after a smoothing process.
Since the stress series detected from microblogs were discrete points,
the loess method was adopted to highlight characteristics of the stress curve (see \ref{alg:alg1}).
In the second step, applying the Poisson based statistical method~\citep{Li2017Analyzing},
we judged whether each candidate interval was a confidential stressful interval.
Finally, we divided the stressful intervals into two sets: the SI set and the U-SI set,
according to its temporal order with neighboring positive events (see \ref{alg:alg2}).
\subsection{Measures}

To extract the restoring patterns \bm{${A}$} for each type of positive events,
we describe a teen's positive and stressful behavioral measures in SI and U-SI sets from three aspects:
posting behavior, stress intensity, and linguistic expressions.

\textbf{Posting behavior}.
Stress could lead to a teen's abnormal posting behaviors,
reflecting the teen's changes in social engagement activity.
For each stressful interval,
we consider four measures of posting behaviors in each time unit (day),
and present each measure as a corresponding series.
The first measure is \emph{posting frequency},
representing the total number of posts per day.
Research in \cite{Li2017Analyzing} indicates that overwhelmed teens usually tend to post more to express their stress for releasing
and seeking comfort from friends.
Further, the second measure \emph{stressful posting frequency} per day
is based on previous stress detection result and highlights the stressful posts among all posts.
Similarly, the third measure is the \emph{positive posting frequency}, indicating the number of positive posts per day.
The forth measure \emph{original frequency} is the number of original posts, which filters out re-tweet and shared posts.
Compared to forwarded posts, original posts indicate higher probability that teens are talking about themselves.
Thus for each day in current interval, the teen's posting behavior is represented as a four-dimension vector.

\textbf{Stress intensity}.
We describe the global stress intensity during a stressful interval through four measures:
\emph{sequential stress level, length, RMS,} and \emph{peak}.
Basically, \emph{stress level} per day constructs a sequential measure during a stressful interval,
recording stress values and fluctuation on each time point.
The \emph{length} measures the lasting time of current stressful interval.
As positive events might conduct impact on overwhelmed teens,
and postpone the beginning or promote the end of the stressful interval,
we take the \emph{length} as a factor representing the interval stress intensity.
To quantify the intensity of fluctuations for stress values,
we adopt the \emph{RMS} (root mean square) of stress values through the interval as the third measure.
In addition, the \emph{peak} stress value is also a measure to show the maximal stress value in current interval.

\textbf{Linguistic expressions}.
We extract the teen's positive and stressful expressions from the content of posts in SI and U-SI sets, respectively.
The first linguistic measure is the frequency of \emph{positive word},
which represents the positive emotion in current interval.
The second measure is the frequency of \emph{positive event topic words} in six dimensions,
reflecting the existence of positive events.
Another important factor is wether existing \emph{self-mentioned words} (i.e., \emph{'I','we','my'}).
Self-mentioned words show high probability that the current stressor event and stressful emotion is related to the author,
rather than the opinion about a public event or life events about others.

Except positive-related linguistic descriptions, we also take stressful linguistic characters as measures,
which is opposite with positive measures, while also offers information from the complementary perspective.
The first stressful linguistic measure is the frequency of \emph{stressor event topic words} in five dimensions,
which represents how many times the teen mentioned a stressor event,
indicating the degree of attention for each type of stressor event.
The frequency of \emph{pressure words} is the second stressful linguistic measure,
reflecting the degree of general stress emotion during the interval.
We adopt this measure specifically because in some cases teens post very short tweets with only stressful emotional words,
where topic-related words are omitted.

Next,
based on the posting behavior, stress intensity and linguistic measures from both the stressful and positive views,
we quantify the difference between SI and U-SI sets, thus to measure the impact of positive events.

\begin{figure*}
\centering
\caption{Correlation towards each types of stressor events}
\includegraphics[width=0.8\linewidth]{figs/BOX.eps}%figs/correlation2.eps
\label{fig:correlation}
\end{figure*}

\subsection{Method}
In our problem,
there are two sets of stressful intervals to compare:
the SI set and the U-SI set,
containing stressful intervals unaffected by positive events
and stressful intervals impacted by positive events, respectively.
The basic elements in each set are stressful intervals,
i.e., the sequential stress values in time line,
which are modeled as multi-dimensional points according to the three groups of measures in section \ref{subsec:pattern}.
Thus we formulate this comparison problem as finding the correlation between the two sets of multi-dimension points.
Specifically, we adopt the multivariate two-sample hypothesis testing method
\cite{Li2017Correlating,Johnson2012Applied} to model such correlation.
In this two-sample hypothesis test problem,
the basic idea is judging whether the multi-dimension points (i.e., stressful intervals)
in set SI and set U-SI are under different statistical distribution.
Assuming the data points in SI and U-SI are randomly sampled from distribution $F^{(1)}$ and $F^{(2)}$, respectively,
then the hypothesis is denoted as:
\begin{equation}
H_1: F^{(1)} =F^{(2)} \quad versus \quad \widetilde{H_1}: F^{(1)} \neq F^{(2)}.
\end{equation}

Under such hypothesis,
$H_1$ indicates points in SI and U-SI are under similar distribution,
while $\widetilde{H_1}$ means points in SI and U-SI are under statistically different distributions,
namely positive events have conducted obvious restoring impact on current stressed teen.
Next, we handle this two-sample hypothesis test problem based on both positive and stressful behavioral measures
(i.e., \emph{posting behavior}, \emph{stress intensity} and \emph{linguisitc expressions}),
thus to quantify the restoring patterns of positive events from multi perspectives.

As a classic statistical topic, various algorithms have been proposed to solve the two-sample hypothesis testing problem.
Since each point in the two sets (SI and U-SI) is depicted in multi-dimensions,
here we take the KNN (k nearest neighbors) \cite{Schilling1986Multivariate}
based method to judge the existence of significant difference between SI and U-SI.
For simplify, we use the symbol $A_1$ to represent set SI,
and $A_2$ represent set U-SI,
namely $A_1$ and $A_2$ are two sets composed of stressful intervals.
In the KNN algorithm,
for each point $\ell_{x}$ in the two sets $A_1$ and $A_2$,
we expect its nearest neighbors (\emph{the most similar points}) belonging to the same set of $\ell_x$,
which indicates the difference between the points in the two cases.
The model derivation process is described in detail in the \ref{mod:mod1} part of the appendix.


\subsection{Results}
\paragraph{Restoring Impact of scheduled positive events}
Basically, we focused on four kinds of scheduled positive events:
\emph{practical activity}, \emph{holiday}, \emph{new year party} and \emph{sports meeting}.
For each of the four scheduled positive events,
we quantify the restoring impact and temporal order
based on corresponding SI and U-SI interval sets of the 124 students.
Table \ref{tab:schedule} shows the experimental results,
where 54.52\%, 78.39\%, 63.39\%, 58.74\% significant restoring impact are detected for the four specific scheduled positive events, respectively, with the total accuracy to 69.52\%.
We adopt the commonly used Pearson correlation algorithms to compare with the two sample statistical method in this study.
The Euclidean metric is used to calculate the distance between two $n$ dimension points $X$ and $Y$.
Experimental results show that our knn-based two sample method (denoted as \emph{KTS}) outperforms the baseline method
with the best improvement in \emph{new year party} to 10.94\%,
and total improvement to 6\%.

\begin{table}[H]
\begin{center}
\caption{\small{Quantify the impact of scheduled positive school events using KTS (the knn-based two sample method adopted in this research) and baseline method.}}
\label{tab:schedule}
\resizebox{.48\textwidth}{13mm}{
\small{
\begin{tabular}{lccccc}
\toprule
&	\emph{practical}	&	         	&	\emph{new year}	&	\emph{sports}	&	\emph{}	\\
&	\emph{activity}	&	\emph{holiday}	&	\emph{party}	&	\emph{meeting}	&	\emph{all}	\\
\midrule
Size of U-SI	&	219 	&	339 	&	235 	&	226 	&	1,019 	\\
Pearson         &55.65\%	&	70.97\%	&	56.45\%	&	54.84\%	&	65.32\% \\
KTS             &54.52\%	&	78.39\%	&	63.39\%	&	58.74\%	&	69.52\% \\
\bottomrule
\end{tabular}
}
}
\end{center}
\end{table}

\begin{table*}
\begin{center}
\caption{\small{Monotonous stress intensity changes in U-SI and SI intervals compared with adjacent intervals.}}
%\resizebox{\textwidth}{15mm}{
\small{
\begin{tabular}{l cccccc cccccc} \\\hline\hline
\multirow{2}{1cm}{}
&\multicolumn{2}{c}{School life}
&\multicolumn{2}{c}{Romantic}
&\multicolumn{2}{c}{Peer relationship}
&\multicolumn{2}{c}{Self-cognition}
&\multicolumn{2}{c}{Family life}
&\multicolumn{2}{c}{All types}\\
&U-SI	    &	SI	        &U-SI	    &SI	        &U-SI	   &SI	
&U-SI	    &	SI	        &	U-SI	&SI	        &U-SI	   &SI\\  \hline
\# Interval         &   365	        &	514	        &	536	        &	587	        &128	    &	391	        &	564	           &	609	            &	321	        &	481	        &	1,914	    &2,582	 \\
Front $\rightarrow$ I &	0.7260 	&	0.7879 	&	0.6903 	&	0.7751 	&	0.7422 	&	0.8159 	&	0.7004 	&	0.7767 	&	0.6791 &	0.7796 	&	0.7017 	&   0.7851\\
I $\rightarrow$ rear  &	0.7589 	&	0.7840 	&	0.7463 	&	0.7905 	&	0.7813 	&	0.8261 	&	0.7500 	&	0.7915 	&	0.7414 	&	0.7942 	&	0.7513 	&   0.7955\\ \hline \hline
\end{tabular}}%}}
\label{tab:fontrear}
\end{center}
\end{table*}

The correlation of positive events towards five types of stressor events
are shown using box-plot in Figure \ref{fig:correlation}.
The stress-buffering pattern of positive events
was closely correlated with posting behavior (80.65\%, n=100, SD=1.96),
stress change mode (67.74\%, n=84, SD=2.04) and microblog linguistic expressions (74.19\%, n=92, SD=2.07).
Positive events conduct most intensive stress-buffering impact in 'family life' (83.87\%, n=104, SD=2.72),
followed by 'peer relationships' (71.77\%, n=89, SD=4.04) and 'school life' (67.74\%, n=84, SD=2.71) dimensions.
In addition,
the correlation between the stress-buffering of positive events
and adolescents' stress in 'family life'
exhibits concentrated trend,
with a higher 25th percentile and 75th percentile.
While the correlation values in 'peer relation'
exhibit the highest 75th percentile and the lowest 25th percentile,
showing a relatively random and unstable stress-buffering impact.

%     school life	romantic	peer relationship	self-cognition	family life
%      40	60	35	59	20
%n     84	64	89	65	104
%SD    2.713	1.728	4.035	1.845	2.721
%ratio 0.6774	0.5161	0.7177	0.5242	0.8387
\section{Study3: Test the dynamic process of stress-buffering function from adolescents' microblogs}
\subsection{Method}
\label{sec:temporal}
To measure the temporal order of stress changes in the two sets of intervals (SI and U-SI) ,
we further compare each interval with the front and rear adjacent intervals, respectively.
Here we adopt the t-test method as the intensity computation function,
to observe whether the occurrence of positive events relieve the monotonic negative effect and the monotonic positive effect.
Details are presents in part \ref{mod:mod2} of the appendix.

\subsection{Result}
\paragraph{Monotonous stress changes caused by positive events}
Further more,
to verify the monotonous stress changes when an positive event impacts a stressful interval,
we collected 1,914 stressful intervals in U-SI,
and 2,582 stressful intervals impacted by positive events in SI.
For each stressful interval in SI and U-SI,
we quantify its stress intensity by comparing with the front and rear adjacent intervals, respectively.
Here four situations are considered and compared according to the temporal order in Section \ref{sec:temporal},
as shown in Table \ref{tab:fontrear},
where the \emph{ratio of intervals} detected with monotonous increase from the \emph{front interval} to \emph{stressful interval} (denoted as \emph{front$ \rightarrow$ I}),
and monotonous decrease from the \emph{stressful interval} to the \emph{rear interval} (denoted as \emph{I $\rightarrow$ rear}) are listed.
Under the impact of positive events,
both the ratio of intensive stress increase in \emph{front$ \rightarrow$ I}
and the ratio of intensive stress decrease in \emph{I $\rightarrow$ rear} are decreased,
showing the effectiveness of the two sample method for quantifying the impact of positive events,
and the rationality of the assumption that positive events could help ease stress of overwhelmed teens.

